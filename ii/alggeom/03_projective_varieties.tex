We will construct the projective space \( \mathbb P^n \), which will be an upgrade to \( \mathbb A^n \); it is not immediately obvious why \( \mathbb P^n \) is considered `better'.
Projective space has some interesting properties, such as:
\begin{itemize}
    \item every pair of lines in \( \mathbb P^2 \) that are distinct meet at a unique point;
    \item if \( V \) is a projective variety (defined shortly) in \( \mathbb P^2 \) defined by a degree \( d \) polynomial, if \( V \) is a manifold then \( V \) is homeomorphic in the Euclidean topology to a closed orientable topological surface of genus \( \binom{d-1}{2} \).
    \item \( \mathbb P^n \) is compact in the Euclidean topology, but \( \mathbb A^n \) is not.
\end{itemize}

\subsection{Definition}
\begin{definition}
    Let \( U \) be a finite-dimensional complex vector space.
    The \emph{projectivisation} of \( U \), written \( \mathbb P(U) \), is the set of lines in \( U \) through the origin \( \vb 0 \in U \).
    Define \( \mathbb P^n = \mathbb P(\mathbb C^{n+1}) \).
\end{definition}
We usually index the coordinates on \( \mathbb C^{n+1} \) with indices \( 0, \dots, n \).
A line in \( \mathbb C^{n+1} \) is therefore given by \( \qty{(a_0 t, \dots, a_n t) \mid t \in \mathbb C} \), and is written \( L_{(a_0, \dots, a_n)} \), where not all \( a_i \) are zero.
We write \( (a_0 : a_1 : \dots : a_n) \) for the corresponding element of \( \mathbb P^n \).
Therefore,
\[ \mathbb P^n = \faktor{\qty{(a_0, \dots, a_n) \mid a_i \in \mathbb C, \text{not all } a_i = 0}}{\text{scaling by } \mathbb C^\star} \]
For example, \( (2 : 1 : -2) = (4 : 2 : -4) \in \mathbb P^2 \).

We can decompose \( \mathbb P^1 \) as
\begin{align*}
        \qty{(a_0 : a_1) \mid a_0 \neq 0} \cup \qty{(a_0 : a_1) \mid a_0 = 0} &= \qty{\qty(1 : z) \mid z \in \mathbb C} \cup \qty{(0 : 1)} \\
        &= \mathbb A^1 \cup\, \text{a point at infinity}
\end{align*}
More generally,
\[ \mathbb P^n = \qty{(a_0 : \dots : a_n) \mid a_0 \neq 0} \cup \qty{(0 : a_1 : \dots : a_n)} = \mathbb A^n \amalg \mathbb P^{n-1} \]
By induction, \( \mathbb P^n = \mathbb A^n \cup \mathbb A^{n-1} \cup \dots \cup \mathbb A^1 \cup \) a point, where the terms other than \( \mathbb A^n \) are considered `at infinity'.
\begin{definition}
    The \emph{Zariski} (respectively \emph{Euclidean}) topology on projective space is the quotient topology for the subspace topology for the Zariski (respectively Euclidean) topology on \( \mathbb C^{n+1} \setminus \qty{\vb 0} \), where \( \mathbb P^n = \faktor{\mathbb C^{n+1} \setminus \qty{\vb 0}}{\sim} \) and \( \mathbb C^{n+1} \setminus \qty{0} \subseteq \mathbb C^{n+1} \).
\end{definition}
There is a copy of \( S^{2n+1} \) inside \( \mathbb C^{n+1} \setminus \qty{\vb 0} \), which therefore surjects onto \( \mathbb P^n \).
\begin{corollary}
    \( \mathbb P^n \) is compact.
\end{corollary}
\begin{proof}
    It is the continuous image of the compact set \( S^{2n+1} \).
\end{proof}
\begin{definition}
    For \( 0 \leq j \leq n \), we define the \emph{\( j \)th coordinate hyperplane} is the set \( H_j = \qty{(\vb a_i) \mid a_j = 0} \subseteq \mathbb P^n \).
\end{definition}
We can naturally identify \( H_j \) with \( \mathbb P^{n-1} \).
\begin{definition}
    The \emph{\( j \)th standard affine patch} \( U_j \) is the complement of \( H_j \).
\end{definition}
There is a natural bijection \( U_j \to \mathbb A^n \) by mapping \( (a_0 : \dots : a_n) \) to \( \qty(\frac{a_0}{a_j}, \dots, \widehat{\frac{a_j}{a_j}}, \dots, \frac{a_n}{a_j}) \) where the hat denotes `forgetting' that element of the tuple.
The inverse function maps \( (b_1, \dots, b_n) \) to \( (b_1 : \dots : b_{j-1} : 1 : b_j : \dots : b_n) \).
We observe that \( \qty{U_j}_{j=0}^n \) is an open cover of \( \mathbb P^n \) in the Zariski topology.

\subsection{Projective varieties}
\begin{example}
    Consider the polynomial \( X_0 + 1 \in \mathbb C[X_0, X_1] \).
    Note that \( X_0 + 1 \) does not define a function on \( \mathbb P^1 \).
    For example, \( (-1 : 0) = (1 : 0) \), but \( X_0 + 1 \) vanishes on the first representative and not the second.
    The vanishing locus of \( X_0 + 1 \) on \( \mathbb P^1 \) is therefore undefined.
    Therefore, we need a slightly more subtle definition of a variety in projective space.
\end{example}
\begin{definition}
    A \emph{monomial} in \( \mathbb C[\vb X] = \mathbb C[X_0, \dots, X_n] \) is an element of the form \( X_0^{d_0} X_1^{d_1} \dots X_n^{d_n} \) where \( d_i \geq 0 \).
    A \emph{term} is a nonzero multiple of a monomial.
    The \emph{degree} of a term \( cX_0^{d_0} \dots X_n^{d_n} \) is \( \sum_{i=0}^n d_i \).
    A \emph{homogeneous polynomial} of degree \( d \) is a finite sum of terms of degree \( d \).
\end{definition}
Any polynomial can be uniquely decomposed as a sum of homogeneous polynomials of different degree; we write \( f = \sum_{i=0}^\infty f_{[i]} \) where the \( f_{[i]} \) are homogeneous of degree \( i \).
Note that this sum is always finite.
\begin{lemma}
    Let \( f \in \mathbb C[\vb X] \) be homogeneous, and let \( (a_0, \dots, a_n) \in \mathbb C^{n+1} \setminus \qty{\vb 0} \).
    Then, if \( f(\vb a) = 0 \), we have \( f(\lambda \vb a) = 0 \) for all \( \lambda \in \mathbb C^\star \).
\end{lemma}
\begin{proof}
    Trivial by checking the definitions.
\end{proof}
\begin{corollary}
    Let \( f \in \mathbb C[\vb X] \) be homogeneous.
    Then
    \[ \mathbb V(f) = \qty{P \in \mathbb P^n \mid f(\vb a) = 0 \text{ for any (or every) representative of } P} \]
    is well-defined.
\end{corollary}
\begin{definition}
    An ideal \( I \trianglelefteq \mathbb C[\vb X] \) is called \emph{homogeneous} if it can be generated by homogeneous polynomials (of potentially different degrees).
\end{definition}
\begin{lemma}
    Let \( I \trianglelefteq \mathbb C[\vb X] \) be an ideal.
    Then \( I \) is homogeneous if and only if whenever \( f \in I \), all of the homogeneous parts \( f_{[r]} \) are also contained in \( I \).
\end{lemma}
\begin{proof}
    Suppose \( I \) is homogeneous.
    Then let \( g_j \) be homogeneous generators of \( I \) of degree \( d_j \).
    Writing \( f = \sum h_j g_j \) for arbitrary \( h_j \in \mathbb C[\vb X] \), we can split each \( h_j \) into its pieces \( h_{j[r]} \).
    Now, \( h_{j[r]} g_j \in I \) is homogeneous, and its degree is \( rd_j \).
    Hence, \( f_{[r]} = \sum_j h_{j[r-dj]} g_j \in I \) as required.
    The converse is trivial by decomposing the generators of the ideal.
\end{proof}
\begin{definition}
    Let \( I \trianglelefteq \mathbb C[\vb X] \) be a homogeneous ideal.
    Then, the \emph{vanishing locus} is \( \mathbb V(I) = \qty{P = (\vb a_i) \in \mathbb P^n \mid \forall f \in I,\, f((\vb a_i)) = 0} \).
    A \emph{projective variety} in \( \mathbb P^n \) is any set of this form.
\end{definition}
Note that we could have defined the vanishing locus using the quantifier `for all \emph{homogeneous} \( f \in I \)'.
\begin{example}
    Let \( U \subseteq \mathbb C^{n+1} \) be any vector subspace.
    Let the projectivisation of \( U \) is a subset of \( \mathbb P^n \), and is a projective variety.
    More concretely, \( U = \qty{\vb v \in \mathbb C^{n+1} \mid \forall j,\, \sum_{i=0}^n a_i^{(j)} v_i = 0} \) for a subset \( \vb a^{(j)} = (a_0^{(j)}, \dots, a_n^{(j)}) \), as a vector subspace is the kernel of some linear map.
    Therefore, \( \mathbb P(U) = \mathbb V(I) \) where \( I \) is the ideal generated by \( F_j = \sum_i a^{(j)}_i X_i \in \mathbb C[\vb X] \).
    More generally, a projective linear space is the projectivisation of a linear subspace.
    Hence, projective linear spaces in \( \mathbb P^n \) are in bijection with linear subspaces in \( \mathbb C^{n+1} \).
\end{example}
\( GL_{n+1}(\mathbb C) \) acts on \( \mathbb P^n \) coordinatewise.
The normal subgroup of scalar matrices \( \mathbb C^\star \subseteq GL_{n+1}(\mathbb C) \) acts trivially on \( \mathbb P^n \).
The quotient is written \( PGL_n(\mathbb C) = \faktor{GL_{n+1}(\mathbb C)}{\mathbb C^\star} \), and acts transitively on \( \mathbb P^n \).
\begin{example}
    The \emph{Segre surface} is the hypersurface \( S_{11} = \mathbb V(X_0X_3 - X_1X_2) \subseteq \mathbb P^3 \).
    Consider the map \( \sigma_{11} \colon \mathbb P^1 \times \mathbb P^1 \to \mathbb P^3 \) given by \( \sigma_{11}((a_0 : a_1), (b_0 : b_1)) = (a_0 b_0 : a_0 b_1 : a_1 b_0 : a_1 b_1) \).
    One can show that this map is well-defined, and in fact, \( \Im \sigma_{11} = S_{11} \).

    First, consider the map \( \mathbb C^2 \times \mathbb C^2 \to \mathbb C^4 \) where we identify \( \mathbb C^4 \) with the space of \( 2 \times 2 \) matrices on \( \mathbb C \), given by the outer product.
    More precisely, \( (v,w) \mapsto vw^\transpose \).
    The image of this map is precisely the set of matrices of rank at most 1.
    Hence, the image is the vanishing locus of \( X_0 X_3 - X_1 X_2 \), the determinant of such a matrix.
\end{example}

\subsection{Homogenisation and projective closure}
Recall that \( \mathbb P^n = U_0 \cup \dots \cup U_n \), where \( U_i = \mathbb P^n \setminus \mathbb V(X_i) \).
We therefore have the following different descriptions of a Zariski topology on \( \mathbb P^n \):
\begin{enumerate}
    \item the quotient of the subspace of the Zariski topology on \( \mathbb C^{n+1} \);
    \item define that \( V \) is Zariski-closed if and only if \( V = \mathbb V(I) \) where \( I \vartriangleleft \mathbb C[\vb X] \) is homogeneous;
    \item the gluing topology: define that a subset \( Z \subseteq \mathbb P^n \) is closed if \( Z \cap U_i \) is closed for all \( i \), as the \( U_i \) are isomorphic to \( \mathbb A^n \).
\end{enumerate}
These three constructions coincide.

If \( V \subseteq \mathbb P^n \) is a projective variety, consider \( U_0 \cap V \subseteq U_0 \).
If \( V = \mathbb V(I) \), then \( U_0 \cap V = \mathbb V(I_0) \) where \( I_0 = \qty{f = F(1, Y_1, \dots, Y_n) \mid F \in I \text{ homogeneous}} \subseteq \mathbb C[Y_1, \dots, Y_n] \).
Identifying \( U_0 \) with \( \mathbb A^n \) with coordinates \( Y_1, \dots, Y_n \) (so \( Y_j = \frac{X_j}{X_0} \)), \( V \cap U_0 \) is naturally an affine variety.

Conversely, let \( W \subseteq \mathbb A^n \) be an affine variety, and identify \( \mathbb A^n \) with \( U_0 \).
Then, the Zariski closure \( \overline W \) of \( W \) inside \( \mathbb P^n \) is a projective variety.
We are interested in studying the precise projective varieties obtained in this way.
\begin{definition}
    Let \( f \in \mathbb C[Y_1, \dots, Y_n] \) be an arbitrary polynomial of total degree \( d \).
    The \emph{homogenisation} of \( f \), written \( F \) or \( f^h \), is
    \[ f^h(X_0, \dots, X_n) = X_0^d f\qty(\frac{X_1}{X_0}, \dots, \frac{X_n}{X_0}) \in \mathbb C[X_0, \dots, X_n] \]
    This is homogeneous of degree \( d \).
    Similarly, if \( I \) is an ideal in \( \mathbb C[Y_1, \dots, Y_n] \), its homogenisation \( I^\star = I^h \) is the ideal generated by the homogenisation of the elements \( f \in I \); this is a homogeneous ideal in \( \mathbb C[X_0, \dots, X_n] \).
    Given an affine variety \( V \subseteq \mathbb A^n \), the \emph{projective closure} of \( V \) is \( \mathbb V(I(V)^h) \subseteq \mathbb P^n \).
\end{definition}
\begin{example}
    Let \( f(Y_1, Y_2) = 1 + Y_1^2 + Y_1 Y_2^2 \).
    Its homogenisation is \( f^h(X_0, X_1, X_2) = X_0^3 + X_0 X_1^2 + X_1 X_2^2 \).
\end{example}
\begin{remark}
    Let \( I = (f_1, \dots, f_r) \subseteq \mathbb C[Y_1, \dots, Y_n] \), and let \( J = (f_1^h, \dots, f_r^h) \).
    Typically, \( J \neq I^h \).
    If \( I \) is principal, this holds: \( I = (f) \) implies \( I^h = (f^h) \).
\end{remark}
\begin{proposition}
    Let \( V \subseteq \mathbb A^n \) be an affine variety.
    Then, the Zariski closure \( \overline V \subseteq \mathbb P^n \) given by identifying \( U_0 = \mathbb A^n \) coincides with the projective closure \( \mathbb V(I(V)^h) \subseteq \mathbb P^n \).
\end{proposition}
\begin{proof}
    Let \( I \) be an ideal in \( \mathbb C[Y_1, \dots, Y_n] \), and let \( V = \mathbb V(I) \).
    Let \( \overline V \) be the Zariski closure.
    Let \( I^h \) be the homogenisation of the ideal.
    Then \( \mathbb V(I^h) \) is Zariski closed, and contains \( V \).
    We will show that this is the smallest such set.

    Suppose \( Y \supseteq V \) is closed, so \( Y = \mathbb V(I') \) where \( I' \) is homogeneous.
    Any homogeneous element in \( I' \) can be written as \( X_0^d f^h \) for some \( f \in \mathbb C[Y_1, \dots, Y_n] \).
    Now, \( X_0^d f^h = 0 \) on \( V \subseteq \mathbb P^n \), so \( f = 0 \) on \( V \subseteq \mathbb A^n \).
    Hence \( f \in I(V) = \sqrt{I} \) by the Nullstellensatz.
    So \( f^m \in I \) for some \( m > 0 \), so \( (f^m)^h = (f^h)^m \in I^h \).
    Hence \( f^h \in \sqrt{I^h} \), so \( X_0^d f^h \in \sqrt{I^h} \).
    Therefore, \( I' \subseteq \sqrt{I^h} \).
\end{proof}
\begin{remark}
    Let \( V \subseteq \mathbb P^n \), and let \( W = V \cap U_0 \subseteq \mathbb A^n \).
    Then \( \overline W \subseteq \mathbb P^n \) is not in general equal to \( V \).
    For example, let \( V = \mathbb V(X_0) \), so \( W = \varnothing \) and \( \overline W = \varnothing \).
    This ambiguity arises due to the \( X_0^d \) term required in the above proof when dehomogenising a polynomial.
\end{remark}
This shows that the topological notion of the Zariski closure and the algebraic notion of the projective closure agree.
\begin{example}
    Let \( V \subseteq \mathbb P^2 \) be given by \( \mathbb V(X_0 X_1 - X_2^2) \).
    We obtain \( V_0 \subseteq U_0 \) given by setting \( X_0 = 1 \), \( V_1 \subseteq U_1 \) given by setting \( X_1 = 1 \), and \( V_2 \subseteq U_2 \) given by setting \( X_2 = 1 \).
    We find \( V_0 = \mathbb V(Y_1 - Y_2^2) \) which is a parabola, and \( V_1 \) is similar.
    \( V_2 = \mathbb V(X_0 X_1 - 1) \), which is a rectangular hyperbola.
\end{example}
\begin{theorem}
    Let \( Q \subseteq \mathbb P^n \) be given by \( \mathbb V(f) \) where \( f \) is a homogeneous quadratic polynomial.
    Then, after a change of coordinates \( A \in PGL_n(\mathbb C) \), \( Q \) has the form \( \mathbb V(X_0^2 + \dots + X_r^2) \) where \( r \) is the rank of the quadratic form \( f \).
\end{theorem}
\begin{proof}
    Use the results from IB Linear Algebra.
\end{proof}
\begin{theorem}[projective Nullstellensatz]
    If \( \mathbb V(I) = \varnothing \subseteq \mathbb P^n \) where \( I \) is a homogeneous ideal, then \( I \supseteq (X_0^m, \dots, X_n^m) \) for some \( m \in \mathbb N \).
    Further, if \( V = \mathbb V(I) \neq \varnothing \), then \( I^h(V) = \sqrt{I} \), where \( I^h(V) \) is the ideal generated by homogeneous polynomials vanishing on \( V \).
\end{theorem}
\begin{proof}
    We reduce to the affine case.
    Let \( I \) be a homogeneous ideal, and let \( V^a = \mathbb V(I) \subseteq \mathbb A^{n+1} \).
    Note that \( \vb 0 \in V^a \), assuming \( V \neq \varnothing \).
    Then there is a continuous map \( V^a \setminus \qty{0} \to V \) obtained by the restriction of \( \mathbb A^{n+1} \setminus \qty{\vb 0} \to \mathbb P^n \).
    Moreover, this map is surjective, so is a quotient map.
    Note that \( V \) is empty if and only if \( V^a = \qty{\vb 0} \).
    So the result holds by the affine Nullstellensatz.
    The second part is similar.
\end{proof}
Let \( V \) be a projective variety in \( \mathbb P^n \).
If \( W \subseteq V \) is a variety closed in \( V \), we say \( W \) is a \emph{closed subvariety} of \( V \).
The complement \( V \setminus W \) is an \emph{open subvariety}.
The closed (respectively open) subvarieties of \( V \) satisfy the axioms of the closed (open) sets of a topology.
We say \( V \) is irreducible if \( V \) cannot be written as \( V_1 \cup V_2 \) for proper closed subvarieties \( V_1, V_2 \).
\begin{proposition}
    \begin{enumerate}
        \item Every projective variety is a finite union of irreducible varieties.
        \item \( V \) is irreducible if and only if \( I^h(V) \) is prime.
    \end{enumerate}
\end{proposition}
\begin{proof}
    Part (i) is identical to the affine case.
    For part (ii), first observe that if \( I \) is a homogeneous ideal which is not prime, we can find homogeneous \( F, G \not\in I \) such that \( FG \in I \), as \( I \) is generated by homogeneous elements.
    Then the proof for the affine case works as before.
\end{proof}
Let \( S \subseteq V \) be a subset.
\( S \) is Zariski dense in \( V \) if and only if every homogeneous polynomial that vanishes on \( S \) vanishes on \( V \).
\begin{proposition}
    Let \( V \subseteq \mathbb P^n \) be an irreducible projective variety.
    Let \( W \subsetneq V \) be a proper closed subvariety.
    Then, \( V \setminus W \) is dense in \( V \).
\end{proposition}
Intuitively, \( W \) is lower-dimensional than \( V \), and \( V \) with a lower-dimensional set removed is dense.
\begin{proof}
    Let \( f \in \mathbb C[\vb X] \) be a homogeneous polynomial that vanishes on \( V \setminus W \).
    As \( W \neq V \), there exists a polynomial \( g \in I^h(W) \setminus I^h(V) \) by the projective Nullstellensatz.
    Then, \( fg \) vanishes on all of \( V \).
    But \( I^h(V) \) is prime as \( V \) is irreducible, so \( f \in I^h(V) \).
\end{proof}

\subsection{Rational functions}
Homogeneous polynomials have well-defined zero sets in \( \mathbb P^n \), but not a well-defined value.
Therefore, we cannot define a coordinate ring \( \mathbb C[V] \) in an analogous way.
However, a ratio of homogeneous polynomials of the same degree does have a well-defined value on \( \mathbb P^n \) away from the vanishing locus of the denominator.
\begin{definition}
    Let \( V \subseteq \mathbb P^n \) be an irreducible projective variety.
    The \emph{function field} or \emph{field of rational functions} is
    \[ \mathbb C(V) = \faktor{\qty{\frac{F}{G} \midd F, G \in \mathbb C[\vb X] \text{ homogeneous and have the same degree, } G \not\in I^h(V)}\ }{\sim} \]
    where \( \frac{F_1}{G_1} \sim \frac{F_2}{G_2} \) if \( F_1 G_2 - F_2 G_1 \in I^h(V) \).
\end{definition}
\begin{lemma}
    The relation \( \sim \) is an equivalence relation.
\end{lemma}
\begin{proof}
    Reflexivity and symmetry are clear.
    Now suppose that \( \frac{F_1}{G_1} \sim \frac{F_2}{G_2} \sim \frac{F_3}{G_3} \), so \( F_2 G_1 - F_1 G_2 \in I^h(V) \) and \( F_3 G_2 - F_2 G_3 \in I^h(V) \).
    Consider \( F_1 G_3 - F_3 G_1 \).
    Multiplying by \( G_2 \), \( F_1 G_2 G_3 - F_3 G_1 G_2 \).
    Since \( G_2 \not\in I^h(V) \), primality of \( I^h(V) \) implies that it suffices to show \( F_1 G_2 G_3 - F_3 G_1 G_2 \in I^h(V) \).
    In the ring \( \faktor{\mathbb C[\vb X]}{I^h(V)} \), we have relations \( F_1 G_2 = F_2 G_1 \) and \( F_3 G_2 = F_2 G_3 \).
    Hence \( F_1 G_2 G_3 - F_3 G_1 G_2 = 0 \) in \( \faktor{\mathbb C[\vb X]}{I^h(V)} \).
\end{proof}
Note that \( \mathbb C(V) \) is a field.
\begin{proposition}
    The field \( \mathbb C(V) \) is a finitely generated field extension of \( \mathbb C \).
\end{proposition}
Note that \( \mathbb C(t) \) is finitely generated as a field, but not finitely generated as a \( \mathbb C \)-module or a \( \mathbb C \)-algebra.
\begin{proof}
    Suppose \( V \neq \varnothing \).
    Then, there is some coordinate function \( X_i \) that is nonzero on \( V \); without loss of generality let \( i = 0 \).
    We claim that \( \frac{X_1}{X_0}, \dots, \frac{X_n}{X_0} \) generate \( \mathbb C(V) \) over \( \mathbb C \).
    Explicitly, if \( \frac{F}{G} \) is a degree 0 ratio, it can be written in terms of the \( \frac{X_j}{X_0} \) and the field operations.
    It suffices to show the result holds when \( \frac{F}{G} \) is of the form \( \frac{M}{G} \) where \( M \) is a monomial.
    Then, it suffices to show the result for \( \frac{G}{M} \) where \( M \) is a monomial by taking reciprocals.
    Hence, it suffices to show the result for \( \frac{M}{M'} \) where \( M, M' \) are monomials, and this is trivial.
\end{proof}
\begin{corollary}
    Let \( V \subseteq \mathbb P^n \) be an irreducible projective variety, not contained in the hyperplane \( \qty{X_0 = 0} \).
    Let \( V_0 = V \cap U_0 \), where \( U_0 \simeq \mathbb A^n \) is the first affine patch.
    Then, \( \mathbb C(V_0) = \mathbb C(V) \), where \( \mathbb C(V_0) = FF(\mathbb C[V_0]) \).
\end{corollary}
\begin{proof}
    \( V_0 \) has coordinate ring
    \[ \faktor{\mathbb C\qty[\frac{X_1}{X_0}, \dots, \frac{X_n}{X_0}]\ }{I(V_0)} \]
    Hence, \( \mathbb C(V_0) = FF(\mathbb C[V_0]) \) is generated by the \( \frac{X_j}{X_0} \).
\end{proof}
\begin{definition}
    Let \( \varphi \in \mathbb C(V) \) and let \( P \in V \).
    We say that \( \varphi \) is \emph{regular} or \emph{defined} at \( P \) if \( \varphi \) can be expressed as \( \frac{F}{G} \) where \( F, G \) are homogeneous of the same degree with \( G(P) \neq 0 \).
    There is a partial function from the set of regular points of \( V \) to \( \mathbb C \).
\end{definition}
\begin{definition}
    The \emph{local ring} of \( V \) at \( P \), written \( \mathcal O_{V,P} \), is the set of \( \varphi \in \mathbb C(V) \) such that \( \varphi \) is regular at \( P \).
    This is a subring of \( \mathbb C(V) \), which is a local ring in the sense of commutative algebra.
\end{definition}
\begin{proposition}
    Let \( V \subseteq \mathbb P^n \) be an irreducible projective variety not contained in \( \qty{X_0 = 0} \).
    Let \( V_0 = V \cap U_0 \) where \( U_0 = \qty{X_0 = 0} \).
    Let \( P \) be a point in \( V_0 \).
    Then, there is a natural isomorphism \( \mathcal O_{V,P} \to \mathcal O_{V_0,P} \) respecting the isomorphism \( \mathbb C(V) \simeq \mathbb C(V_0) \).
\end{proposition}
\begin{proof}
    Follows by unfolding the definitions.
\end{proof}

\subsection{Rational maps}
% We denote a \emph{rational map} using notation such as \( \varphi \colon \mathbb P^n \dashrightarrow \mathbb P^m \).
Let \( F_0, \dots, F_m \in \mathbb C[\vb X] = \mathbb C[X_0, \dots, X_n] \) be homogeneous of the same degree \( d \).
Define \( \vb F = (F_0, \dots, F_m) \colon \mathbb C^{n+1} \to \mathbb C^{m+1} \).
\begin{proposition}
    The map \( \vb F \) descends to a well-defined map of sets \( \varphi \colon \mathbb P^n \setminus \bigcap_j \mathbb V(F_j) \to \mathbb P^m \).
    If \( P \) is represented by \( \vb a = (a_0, \dots, a_n) \), then \( \varphi(P) \) is represented by \( (F_0(\vb a), \dots, F_m(\vb a)) \).
\end{proposition}
\begin{proof}
    Since all \( F_j \) are homogeneous of the same degree \( d \), \( \lambda \vb a = (\lambda a_0, \dots, \lambda a_n) \) gives
    \[ (F_0(\lambda\vb a), \dots, F_m(\lambda \vb a)) = \lambda^d (F_0(\vb a), \dots, F_m(\vb a)) \]
    which is equivalent to \( \varphi(P) \).
\end{proof}
We will denote such maps \( \vb F = (F_0, \dots, F_m) \) by \( \varphi \colon \mathbb P^n \dashrightarrow \mathbb P^m \).

Let \( G \) be a nonzero homogeneous polynomial in \( X_0, \dots, X_n \).
Given \( \vb F \colon \mathbb P^n \dashrightarrow \mathbb P^m \), we can also consider \( G\vb F = (GF_0, \dots, GF_n) \colon \mathbb P^n \dashrightarrow \mathbb P^m \).
Observe that the maps \( \vb F \) and \( G\vb F \) have different domains, but coincide at points where they are both defined.
Note that there is a `best' representative \( \vb F \), as \( \mathbb C[\vb X] \) is a unique factorisation domain, but we will not use this notion here, because not all rings that we will use are unique factorisation domains.
\begin{definition}
    Let \( V \subseteq \mathbb P^n \) be an irreducible projective variety.
    Let \( F_0, \dots, F_m \) be homogeneous polynomials in \( \mathbb C[X_0, \dots, X_n] \) of fixed degree \( d \), and not all contained in \( I^h(V) \).
    They determine a map of sets \( V \setminus \bigcap_j \mathbb V(F_j) \to \mathbb P^n \) by the previous construction.
    Two such tuples \( (F_0, \dots, F_m) \) and \( (G_0, \dots, G_m) \) are said to \emph{determine the same map} if \( F_i G_j - F_j G_i \in I^h(V) \).
    A \emph{rational map} from \( V \) to \( \mathbb P^m \) is an equivalence class of tuples \( (F_0, \dots, F_m) \) as above, where two tuples are equivalent if they determine the same map.
\end{definition}
\begin{definition}
    A point \( P \in V \) is a \emph{regular point} of a rational map \( \varphi \colon V \dashrightarrow \mathbb P^n \) if there is a representative \( (F_0, \dots, F_m) \) of \( \varphi \) such that \( F_j(P) \neq 0 \) for some \( j \).
    The \emph{domain} of \( \varphi \) is the set of regular points of \( \varphi \).
    A rational map \( \varphi \) is called a \emph{morphism} if the domain of \( \varphi \) is \( V \); in this case, we write \( V \to \mathbb P^m \).
\end{definition}
\begin{example}
    A linear map \( \varphi \colon \mathbb P^n \dashrightarrow \mathbb P^m \) is given by an \( (m + 1) \times (n + 1) \) matrix \( A = (a_{ij}) \).
    Concretely, we can define \( \varphi = (F_0, \dots, F_m) \) where \( F_j = \sum_i a_{ij} X_i \).
    If \( A \) has rank \( n + 1 \leq m + 1 \), then \( \varphi \) is a morphism.
\end{example}
\begin{example}[projection from a point]
    Let \( P = (0 : 0 : 1) \in \mathbb P^2 \).
    The \emph{projection from \( P \)} is the rational map \( \pi \colon \mathbb P^2 \dashrightarrow \mathbb P^1 \) defined by \( (a_0 : a_1 : a_2) \mapsto (a_0 : a_1) \).
    \( \pi \) is not regular at \( P \), and regular everywhere else.

    Let \( C = \mathbb V(f_d) \) where \( f_d \) is a homogeneous polynomial of degree \( d \).
    Suppose that \( P \not\in C \).
    The composition is therefore a morphism \( \varpi \colon C \to \mathbb P^1 \).
    One can show that for almost all choices of \( Q \in \mathbb P^1 \), the fibre \( \varpi^{-1}(Q) \) is a set of size \( d \).
\end{example}
\begin{example}
    Let \( C = \mathbb V(X_0 X_2 - X_1^2) \subseteq \mathbb P^2 \).
    Consider the projection from \( (0 : 0 : 1) \), and restrict this projection to \( C \) to obtain a map \( \pi \colon C \dashrightarrow \mathbb P^1 \) defined by \( \pi(a_0:a_1:a_2) = (a_0:a_1) \).
    By changing representatives, we can show \( \pi \) is a morphism, even though \( (0 : 0 : 1) \in C \).

    The map \( \pi \) is determined by \( (X_0, X_1) \); we must look for other pairs \( (F_0, F_1) \) that determine the same rational map as \( \pi \), so \( F_0 X_1 - F_1 X_0 \in I^h(C) = (X_0 X_2 - X_1^2) \).
    Notice that this relation is satisfied by \( (X_1, X_2) \), so \( \pi \) agrees with the function \( \pi'(a_0:a_1:a_2) = (a_1:a_2) \) on \( C \).
    So \( \pi \) is regular at \( (0 : 0 : 1) \in C \), so \( \pi \) is a morphism.

    Observe that for \( \pi \colon C \to \mathbb P^1 \), \( \pi^{-1}(q) \) is a single point for \( q \in \mathbb P^1 \).
    One can show also that \( \pi \) is surjective.
\end{example}
If \( W \) is a projective variety, a rational map (or morphism) \( V \to W \) is a rational map (or morphism) \( V \to \mathbb P^m \) with image contained in \( W \).
A morphism \( \varphi \colon V \to W \) is an isomorphism if it has a two-sided inverse morphism.
\begin{proposition}
    Let \( C \) be the vanishing locus of a homogeneous polynomial \( f \in \mathbb C[X_0, X_1, X_2] \) of degree 2 in \( \mathbb P^2 \).
    Then, if \( f \) is irreducible then \( C \simeq \mathbb P^1 \).
\end{proposition}
\begin{proof}
    By changing coordinates, we can assume \( f = X_0 X_2 - X_1^2 \); the rank of the quadratic form is 2 as \( f \) is irreducible.
    By the example above, we have a morphism \( \pi \colon C \to \mathbb P^1 \) by projection from \( (0 : 0 : 1) \).
    We define an inverse map \( \mu \colon \mathbb P^1 \to \mathbb P^2 \) by \( \mu(Y_0 : Y_1) = (Y_0^2 : Y_0 Y_1 : Y_1^2) \).
    The image of \( \mu \) lies in \( C \), and the compositions are inverse.
\end{proof}
There is only one conic in two-dimensional projective space, up to changing coordinates.
\begin{example}[Cremona transformation]
    Consider the rational map \( \mathbb P^2 \dashrightarrow \mathbb P^2 \) given by
    \[ \kappa(X_0 : X_1 : X_2) = (X_1 X_2 : X_0 X_2 : X_0 X_1) \]
    This can be thought of as a coordinatewise reciprocal map.
    The Cremona transformation maps lines into conics.
    Suppose \( \ell \) is a line not given by the vanishing locus of any of the coordinate functions \( X_i \).
    Then, consider the subset \( \kappa(\dom \kappa \cap \ell) \subseteq \mathbb P^2 \); this is the analogue of the image in the case of rational maps.
    One can show that the closure of this set is a conic.
\end{example}
\begin{example}[Veronese embedding]
    Let \( F_0, \dots, F_m \) be the list of monomials of degree \( d \) in \( X_0, \dots, X_n \), so \( m = \binom{n+d}{d} - 1 \).
    We define the \( \nu_d \colon \mathbb P^n \to \mathbb P^m \) mapping \( (\vb a) \) to \( (F_0(\vb a), \dots, F_m(\vb a)) \).
    One can show this is a morphism.
    Note that the map \( \mu(Y_0 : Y_1) = (Y_0^2 : Y_0 Y_1 : Y_1^2) \) used in the previous proposition is an instance of this embedding.
    In general, \( \nu_d \) is injective, and the image of \( \nu_d \) is a projective variety isomorphic to \( \mathbb P^n \).
    This fact has a straightforward but tedious proof.
\end{example}
Note that \( \mathbb P^n \times \mathbb P^m \not\simeq \mathbb P^{n+m} \).
\begin{example}[Segre embedding]
    Let \( n, m > 0 \) be integers.
    The \emph{Segre embedding} is the map \( \sigma_{mn} \colon \mathbb P^m \times \mathbb P^n \to \mathbb P^{mn+m+n} \) defined by \( \sigma_{mn}((x_i), (y_j)) = (x_i y_j) \).
    We label the coordinates of \( \mathbb P^{mn+m+n} \) using \( Z_{ij} \) for \( 0 \leq i \leq m \) and \( 0 \leq j \leq n \).
    Note that \( (m+1)(n+1) - 1 \); we have a map \( U \times V \to U \otimes V \) and then take the projectivisation, giving the correct dimension.
\end{example}
\begin{theorem}
    The map \( \sigma_{mn} \) is a bijection between \( \mathbb P^m \times \mathbb P^n \) and the projective variety \( \mathbb V(I) \) where \( I \) is the ideal generated by the \( Z_{ij} Z_{pq} - Z_{iq} Z_{pj} \).
\end{theorem}
\begin{proof}
    Clearly, \( \sigma_{mn}(\mathbb P^m \times \mathbb P^n) \subseteq V = \mathbb V(I) \).
    Now, consider the affine piece \( V_{00} = V \cap \qty{Z_{00} \neq 0} \subseteq \mathbb A^{mn+m+n} \).
    The inhomogeneous ideal defining \( V_{00} \) is generated by \( Y_{ij} - Y_{i0} Y_{0j} \) where \( 1 \leq i \leq m \) and \( 1 \leq j \leq n \), and \( Y_{ij} = \frac{Z_{ij}}{Z_{00}} \).
    Note that elements \( Y_{ij} Y_{pq} - Y_{iq} Y_{pj} \) for other indices automatically lie in this ideal.
    On this patch, \( \sigma_{mn} \) defines a morphism \( \mathbb A^m \times \mathbb A^n \to \mathbb V(I_{00}) \).
    There is an inverse \( \mathbb A^{mn+m+n} \to \mathbb A^m \times \mathbb A^n \), given by
    \[ (Y_{ij}) \mapsto ((Y_{10}, \dots, Y_{m0}), (Y_{01}, \dots, Y_{0n})) \]
    One can check that this is indeed an inverse; this process can be repeated for all other patches \( \qty{Z_{ij} \neq 0} \), so \( \sigma_{mn} \) is as claimed.
\end{proof}
Hence, if \( V, W \) are projective varieties, \( V \times W \) is naturally also a projective variety.

\subsection{Composition of rational maps}
Let \( \varphi \colon V \dashrightarrow W \) and \( \psi \colon W \dashrightarrow Z \) be rational maps between irreducible varieties.
The composition \( \psi \circ \varphi \) of rational maps may not be well-defined, as the image of the domain of \( \varphi \) could lie entirely inside the locus of indeterminacy of \( \psi \).
\begin{definition}
    A rational map \( \varphi \colon V \dashrightarrow W \) is \emph{dominant} if \( \varphi(\dom \varphi) \) is Zariski dense in \( W \).
\end{definition}
\begin{proposition}
    If \( \varphi \) is dominant, then \( \psi \circ \varphi \) is well-defined for any rational map \( \psi \colon W \dashrightarrow Z \).
\end{proposition}
\begin{proof}
    Let \( U \) denote a dense open set in \( \dom \varphi \), and let \( U' \) be a dense open set in \( \dom \psi \).
    Then, let \( U'' = U \cap \varphi^{-1}(U') \), which is open in \( V \).
    The composition \( \psi \circ \varphi \) is well-defined on \( U'' \).
    This is a rational map as the composition of polynomials is a polynomial.
\end{proof}
\begin{definition}
    If \( \varphi \colon V \dashrightarrow W \) and \( \psi \colon W \dashrightarrow V \) are such that \( \varphi \circ \psi \) and \( \psi \circ \varphi \) are equivalent to the identity map on \( W \) and \( V \) respectively, we say that \( V \) and \( W \) are \emph{birational} and that \( \varphi \) and \( \psi \) are \emph{birational maps}.
\end{definition}
\begin{example}
    Any isomorphism is birational.
\end{example}
\begin{example}
    Consider the Cremona map \( \kappa \colon \mathbb P^2 \dashrightarrow \mathbb P^2 \) defined as above by \( (x_0 : x_1 : x_2) \mapsto (x_1 x_2 : x_0 x_2 : x_0 x_1) \).
    Intuitively, \( (x_0 : x_1 : x_2) \mapsto \qty(\frac{1}{x_0} : \frac{1}{x_1} : \frac{1}{x_2}) \).
    Then \( \kappa \) is self-inverse as a rational map, hence birational.
    It is not an isomorphism as it is not defined everywhere.
\end{example}
\begin{remark}
    One can construct the group \( \mathrm{Bir}(\mathbb P^2) \) of birational automorphisms of \( \mathbb P^2 \).
    This group contains a copy of \( PGL_2(\mathbb C) \) and the subgroup generated by \( \kappa \) above.
\end{remark}
\begin{theorem}
    Let \( V, W \) be irreducible projective varieties.
    Then \( V \) is birational to \( W \) if and only if \( \mathbb C(V) \) and \( \mathbb C(W) \) are isomorphic as fields.
\end{theorem}
Recall the similar result that if \( V, W \) are affine varieties, \( V \) is isomorphic to \( W \) if and only if \( \mathbb C[V] \) and \( \mathbb C[W] \) are isomorphic as \( \mathbb C \)-algebras.
\begin{proof}
    Suppose first that \( V \) is birational to \( W \), so \( \varphi \colon V \dashrightarrow W \) is a birational map.
    Let \( f \in \mathbb C(W) \).
    Then, \( f \) gives a function \( W \dashrightarrow \mathbb A^1 = \mathbb C \), and composition gives a map of fields \( \varphi^\star \colon \mathbb C(W) \to \mathbb C(V) \) defined by \( f \mapsto f \circ \varphi \).
    Similarly, \( \varphi^{-1} \) gives a map \( \mathbb C(V) \to \mathbb C(W) \), and the compositions are identical, so we obtain an isomorphism of fields.

    For the converse, suppose we have \( \mathbb C(V) \simeq \mathbb C(W) \) as fields.
    Suppose that \( V \subseteq \mathbb P^n \) is not contained in \( \qty{X_0 = 0} \), and \( W \subseteq \mathbb P^m \) is not contained in \( \qty{Y_0 = 0} \).
    We have shown that \( \mathbb C(V) = \mathbb C(x_1, \dots, x_n) \) where \( x_i \) is the rational function determined by \( \frac{X_i}{X_0} \).
    Similarly, \( \mathbb C(W) = \mathbb C(y_1, \dots, y_m) \) where \( y_j \) is determined by \( \frac{Y_j}{Y_0} \).

    An isomorphism \( \mathbb C(V) \simeq \mathbb C(W) \) identifies each \( y_j \) with \( f_j(\vb x) \) where \( f_j \) is a rational function in \( n \) variables.
    Writing each \( f_j(\vb x) \) as a rational function in the \( \frac{X_i}{X_0} \), we can clear denominators by multiplying by some polynomial in the \( \frac{X_i}{X_0} \) and homogenise with respect to \( X_0 \).
    We then obtain homogeneous polynomials \( F_0, \dots, F_m \) in \( X_0, \dots, X_n \) such that
    \[ f_j\qty(\frac{X_1}{X_0}, \dots, \frac{X_n}{X_0}) = \frac{F_j(X_0, \dots, X_n)}{F_0(X_0, \dots, X_n)} \]
    Now, \( F_0, \dots, F_m \) determine a rational map \( V \dashrightarrow W \).
    This can be repeated with the \( x_i \) and \( y_j \) reversed to obtain a rational map \( W \dashrightarrow V \).
    One can show that these are inverses.
\end{proof}
