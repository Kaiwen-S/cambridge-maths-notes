\chapter[Algebraic Topology \\ \textnormal{\emph{Lectured in Michaelmas \oldstylenums{2022} by \textsc{Prof.\ J.\ Rasmussen}}}]{Algebraic Topology}
\emph{\Large Lectured in Michaelmas \oldstylenums{2022} by \textsc{Prof.\ J.\ Rasmussen}}

This course is an introduction to the basic ideas of algebraic topology.
In the first half of the course, we study an invariant of based topological spaces called the fundamental group.
This invariant associates a group to a topological space (with a basepoint).
It has the important property that a continuous map between topological spaces induces a homomorphism between their fundamental groups, and that the composition of two maps is mapped to the composition of the corresponding homomorphisms.
In slightly fancier language, the fundamental group determines a functor from the category of based topological spaces to the category of groups.
The phenomena that the fundamental group detects are essentially one-dimensional; it measures the failure of closed loops in the space to bound two-dimensional disks.

In the second half of the course, we study another functor from spaces to groups, called homology, which enables us to understand higher-dimensional `holes' in the space.
There are many different ways to define homology; we use a relatively concrete one called simplicial homology, which makes sense for a somewhat restricted class of spaces.
The notion of homology plays a central role in modern geometry and topology as well as in many branches of algebra and number theory.

Using these invariants we can distinguish various spaces from each other; for example, we can prove that \( \mathbb R^n \) is not homeomorphic to \( \mathbb R^m \) when \( n \) is not equal to \( m \).
We are also able to prove the fundamental theorem of algebra, and to show that certain maps from a space to itself (for example, any continuous map from the closed \( n \)-dimensional disk to itself) must have fixed points.

\subfile{../../ii/algtop/main.tex}
