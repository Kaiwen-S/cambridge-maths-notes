\subsection{Propositions as types}
We will work with the fragment of \( \mathsf{IPC} \), denoted \( \mathsf{IPC}(\to) \), where the only connective is \( \to \), and the deduction rules are \( \to \)-I, \( \to \)-E, \textsc{Ax}.

If \( \mathcal L \) is a propositional language for \( \mathsf{IPC}(\to) \) and \( P \) is its set of primitive propositions, we can generate a simply typed \( \lambda \)-calculus \( \lambda(\to) \) by taking the set of primitive types \( \mathcal U \) to be \( P \).
Then the types \( \Pi \) and the propositions \( \mathcal L \) are generated by the same grammar
\[ \mathcal U \mid \Pi \to \Pi \]
A proposition is thus the type of its proofs, and a context is a set of hypotheses.
\begin{proposition}[Curry--Howard correspondence for \( \mathsf{IPC}(\to) \)]
    Let \( \Gamma \) be a context for \( \lambda(\to) \), and let \( \varphi \) be a proposition.
    Then
    \begin{enumerate}
        \item If \( \Gamma \Vdash M : \varphi \), then
        \[ \abs{\Gamma} = \qty{\tau \in \Pi \mid \exists x.\, (x : \tau) \in \Gamma} \vdash_{\mathsf{IPC}(\to)} \varphi \]
        \item If \( \Gamma \vdash_{\mathsf{IPC}(\to)} \varphi \), then there is a simply typed \( \lambda \)-term \( M \) such that
        \[ \qty{(x_\tau : \tau) \mid \tau \in \Gamma} \Vdash M : \varphi \]
    \end{enumerate}
\end{proposition}
\begin{proof}
    \emph{Part (i).}
    We use induction over the derivation of \( \Gamma \Vdash M : \varphi \).
    If \( x \) is a variable not occurring in \( \Gamma' \), and the derivation is of the form \( \Gamma', x : \varphi \Vdash x : \varphi \), then we must prove that \( \abs{\Gamma', x : \varphi} \vdash \varphi \), and this holds as \( \varphi \vdash \varphi \).

    If the derivation has \( M \) of the form \( \lambda x:\sigma.\, N \) and \( \varphi = \sigma \to \tau \), then we must have that \( \Gamma, x: \sigma \Vdash N : \tau \).
    By the inductive hypothesis, we have \( \abs{\Gamma, x : \sigma} \vdash \tau \), so \( \abs{\Gamma}, \sigma \vdash \tau \).
    Thus we obtain a proof of \( \sigma \to \tau \) from \( \abs{\Gamma} \) by \( \to \)-I.

    If the derivation is of the form \( \Gamma \Vdash (P\, Q) : \varphi \), then we must have \( \Gamma \Vdash P : \sigma \to \varphi \) and \( \Gamma \Vdash Q : \sigma \) for some \( \sigma \).
    By the inductive hypothesis, \( \abs{\Gamma} \vdash \sigma \to \varphi \) and \( \abs{\Gamma} \vdash \sigma \).
    Then the result holds by \( \to \)-E.

    \emph{Part (ii).}
    We use induction over the proof tree of \( \Gamma \vdash_{\mathsf{IPC}(\to)} \varphi \).
    We write
    \[ \Delta = \qty{(x_\tau : \tau) \mid \tau \in \Gamma} \]
    Suppose that we are at a stage of the proof that uses \textsc{Ax}, so \( \Gamma, \varphi \vdash \varphi \).
    If \( \varphi \in \Gamma \), then clearly \( \Delta \Vdash x_\varphi : \varphi \).
    Otherwise, \( \Delta, x_\varphi : \varphi \Vdash x_\varphi : \varphi \) as required.

    Suppose that we are at a stage of the proof that uses \( \to \)-E, so
    \[ \inferrule{\Gamma \vdash \varphi \to \psi \and \Gamma \vdash \varphi}{\Gamma \vdash \psi} \]
    By the inductive hypothesis, there are \( \lambda \)-terms \( M, N \) such that \( \Delta \Vdash M : \varphi \to \psi \) and \( \Delta \Vdash N : \varphi \).
    Then \( \Delta \Vdash (M\, N) : \psi \) as required.

    Finally, suppose we are at a stage of the proof that uses \( \to \)-I, so
    \[ \inferrule{\Gamma, \varphi \vdash \psi}{\Gamma \vdash \varphi \to \psi} \]
    If \( \varphi \in \Gamma \), then by the inductive hypothesis, there is a \( \lambda \)-term \( M \) such that \( \Delta \Vdash M : \psi \).
    By the weakening rule, \( \Delta, x : \varphi \Vdash M : \psi \) where \( x \) is a variable that does not occur in \( \Delta \).
    Then \( \Delta \Vdash (\lambda x:\varphi.\, M) : \varphi \to \psi \) as required.
    Now suppose \( \varphi \notin \Gamma \).
    By the inductive hypothesis we obtain a \( \lambda \)-term \( M \) such that \( \Delta, x_\varphi : \varphi \Vdash M : \psi \).
    Then similarly \( \Delta \Vdash (\lambda x_\varphi:\varphi.\, M) : \varphi \to \psi \).
\end{proof}
This justifies the Brouwer--Heyting--Kolmogorov interpretation of intuitionistic logic.
\begin{example}
    Let \( \varphi, \psi \) be primitive propositions, and consider the \( \lambda \)-term
    \[ \lambda f:(\varphi \to \psi) \to \varphi.\,\lambda g: \varphi \to \psi.\, g(fg) \]
    This term has type
    \[ ((\varphi \to \psi) \to \varphi) \to ((\varphi \to \psi) \to \psi) \]
    The term encodes a proof of this proposition in \( \vdash_{\mathsf{IPC}(\to)} \).
    The corresponding proof tree is
    \[ \inferrule*[right=\( \to \)-I]
        {\inferrule*[right=\( \to \)-I]{\inferrule*[right=\( \to \)-E]{
            \inferrule*[right=\( \to \)-E]{g : [\varphi \to \psi] \and f : [(\varphi \to \psi) \to \varphi]}{fg : \varphi}
        \and g : [\varphi \to \psi]}{g(fg) : \psi}}
        {\lambda g: \varphi \to \psi.\, g(fg) : (\varphi \to \psi) \to \psi}}
        {\lambda f:(\varphi \to \psi) \to \varphi.\,\lambda g: \varphi \to \psi.\, g(fg) : ((\varphi \to \psi) \to \varphi) \to ((\varphi \to \psi) \to \psi)} \]
\end{example}

\subsection{Full simply typed lambda calculus}
The types of the full simply typed \( \lambda \)-calculus \( \mathsf{ST}\lambda\mathsf{C} \) are generated by the following grammar.
\[ \Pi \Coloneq \mathcal U \mid \Pi \to \Pi \mid \Pi \times \Pi \mid \Pi + \Pi \mid 1 \mid 0 \]
where \( \mathcal U \) is a set of primitive types or type variables.
The terms are of the form
\begin{align*}
    \Lambda_\Pi \Coloneq\, & V \mid (\lambda x : \Pi.\, \Lambda_\Pi) \mid \Lambda_\Pi \, \Lambda_\Pi \mid \\
    &\langle \Lambda_\Pi, \Lambda_\Pi \rangle \mid \pi_1(\Lambda_\Pi) \mid \pi_2(\Lambda_\Pi) \mid \\
    &\iota_1(\Lambda_\Pi) \mid \iota_2(\Lambda_\Pi) \mid \mathsf{case}(\Lambda_\Pi;V.\Lambda_\Pi;V.\Lambda_\Pi) \mid \\
    &\star \mid {!_\Pi\, \Lambda_\Pi}
\end{align*}
where \( V \) is an infinite set of variables, and \( \star \) is a constant.
This expanded syntax comes with new typing rules.
\begin{mathpar}
    \inferrule{\Gamma \Vdash M : \psi \times \varphi}{\Gamma \Vdash \pi_1(M) : \psi}
    \and
    \inferrule{\Gamma \Vdash M : \psi \times \varphi}{\Gamma \Vdash \pi_2(M) : \varphi}
    \and
    \inferrule{\Gamma \Vdash M : \psi \and \Gamma \Vdash N : \varphi}{\Gamma \Vdash \langle M, N \rangle : \psi \times \varphi}
    \and
    \inferrule{\Gamma \Vdash M : \psi}{\Gamma \Vdash \iota_1(M) : \psi + \varphi}
    \and
    \inferrule{\Gamma \Vdash M : \varphi}{\Gamma \Vdash \iota_2(M) : \psi + \varphi}
    \and
    \inferrule{\Gamma \Vdash L : \psi + \varphi \and \Gamma, x : \psi \Vdash M : \rho \and \Gamma, y : \varphi \Vdash N : \rho}{\Gamma \Vdash \mathsf{case}(L;x^\psi.M;y^\varphi.N) : \rho}
    \and
    \inferrule{\ }{\Gamma \Vdash \star : 1}
    \and
    \inferrule{\Gamma \Vdash M : 0}{\Gamma \Vdash {!_\varphi}\, M : \varphi}
\end{mathpar}
This typing relation captures the Brouwer--Heyting--Kolmogorov interpretation when paired with new reduction rules.
\begin{mathpar}
    \pi_1(\langle M, N \rangle) \to_\beta M
    \and
    \pi_2(\langle M, N \rangle) \to_\beta N
    \and
    \langle \pi_1(M), \pi_2(M) \rangle \to_\eta M
    \and
    \mathsf{case}(\iota_1(M);x^\psi.K;y^\varphi.L) \to_\beta K[x \coloneq M]
    \and
    \mathsf{case}(\iota_2(M);x^\psi.K;y^\varphi.L) \to_\beta L[y \coloneq M]
    \and
    \text{if } \Gamma \Vdash M : 1 \text{ then } M \to_\eta \star
\end{mathpar}
We can expand propositions-as-types to our new types:
\begin{enumerate}
    \item \( 0 \) corresponds to \( \bot \);
    \item \( 1 \) corresponds to \( \top \);
    \item product types correspond to conjunctions;
    \item coproduct types correspond to disjunctions.
\end{enumerate}
In this way, propositions correspond to types.
Redexes are now those expressions consisting of a constructor (pair formation, \( \lambda \)-abstraction, and injections) followed by the corresponding destructor (projections, applications, and case expressions).
\begin{example}
    Consider the following proof of \( (\varphi \wedge \chi) \to (\psi \to \varphi) \).
    \[ \inferrule{
        \inferrule{
            \inferrule{[\varphi \wedge \chi]}{\varphi} \and [\psi]
        }{\psi \to \varphi}
    }{(\varphi \wedge \chi) \to (\psi \to \varphi)} \]
    Annotating the corresponding \( \lambda \)-terms, we obtain
    \[ \inferrule{
        \inferrule{
            \inferrule{p : [\varphi \wedge \chi]}{\pi_1(p) : \varphi} \and b : [\psi]
        }{\lambda b^\psi.\, \pi_1(p) : \psi \to \varphi}
    }{\lambda p^{\varphi \times \chi}.\, \lambda b^\psi.\, \pi_1(p) : (\varphi \wedge \chi) \to (\psi \to \varphi)} \]
    Hence this proof tree corresponds to the \( \lambda \)-term
    \[ \lambda p^{\varphi \times \chi}.\, \lambda b^\psi.\, \pi_1(p) : (\varphi \times \chi) \to (\psi \to \varphi) \]
\end{example}
In summary, the Curry--Howard correspondence for the whole of \( \mathsf{IPC} \) and \( \mathsf{ST}\lambda\mathsf{C} \) states that
\begin{enumerate}
    \item (primitive) types correspond to (primitive) propositions;
    \item variables correspond to hypotheses;
    \item \( \lambda \)-terms correspond to proofs;
    \item inhabitation of a type corresponds to provability of a proposition;
    \item term reduction corresponds to proof normalisation.
\end{enumerate}

\subsection{Heyting semantics}
Boolean algebras represent truth-values of classical propositions.
We can generalise this notion to intuitionistic logic.
\begin{definition}
    A \emph{Heyting algebra} \( H \) is a bounded lattice equipped with a binary operation \( \Rightarrow : H \times H \to H \) such that
    \[ a \wedge b \leq c \iff a \leq b \Rightarrow c \]
    A \emph{morphism} of Heyting algebras is a function that preserves all finite meets and joins (including true and false) and \( \Rightarrow \).
\end{definition}
In particular, if \( f \) is a morphism of Heyting algebras and \( a \leq b \), then \( f(a) \leq f(b) \).
\begin{example}
    \begin{enumerate}
        \item Every Boolean algebra is a Heyting algebra by defining \( a \Rightarrow b \) to be \( \neg a \vee b \).
        Note that \( \neg a = a \Rightarrow \bot \).
        \item Every topology is a Heyting algebra, where \( U \Rightarrow V = ((X \setminus U) \cup V)^\circ \).
        \item Every finite distributive lattice is a Heyting algebra.
        \item The Lindenbaum--Tarski algebra of a propositional theory \( \mathcal T \) with respect to \( \mathsf{IPC} \) is a Heyting algebra.
        % important exercise
    \end{enumerate}
\end{example}
\begin{definition}
    Let \( H \) be a Heyting algebra and let \( \mathcal L \) be a propositional language with a set \( P \) of primitive propositions.
    An \emph{\( H \)-valuation} is a function \( v : P \to H \), recursively expanded to \( \mathcal L \) by the rules
    \begin{enumerate}
        \item \( v(\bot) = \bot \);
        \item \( v(A \wedge B) = v(A) \wedge v(B) \);
        \item \( v(A \vee B) = v(A) \vee v(B) \);
        \item \( v(A \to B) = v(A) \Rightarrow v(B) \).
    \end{enumerate}
    We say that a proposition \( A \) is \emph{\( H \)-valid} if \( v(A) = \top \) for all valuations \( v \).
    \( A \) is an \emph{\( H \)-consequence} of a finite set of propositions \( \Gamma \) if \( v\qty(\bigwedge \Gamma) \leq v(A) \), and write \( \Gamma \vDash_H A \).
\end{definition}
\begin{lemma}[soundness]
    Let \( H \) be a Heyting algebra and let \( v : \mathcal L \to H \) be an \( H \)-valuation.
    If \( \Gamma \vdash_{\mathsf{IPC}} A \), then \( \Gamma \vDash_H A \).
\end{lemma}
\begin{proof}
    We proceed by induction over the derivation of \( \Gamma \vdash_{\mathsf{IPC}} A \).
    \begin{enumerate}
        \item (\textsc{Ax}) \( v\qty((\bigwedge \Gamma) \wedge A) = v\qty(\bigwedge \Gamma) \wedge v(A) \leq v(A) \).
        \item (\( \wedge \)-I) In this case, \( A = B \wedge C \) and we have derivations \( \Gamma_1 \vdash B, \Gamma_2 \vdash C \) with \( \Gamma_1, \Gamma_2 \subseteq \Gamma \).
        By the inductive hypothesis, \( v(\Gamma_1) \leq v(B) \) and \( v(\Gamma_2) \leq v(C) \), hence
        \[ v\qty(\bigwedge \Gamma) \leq v(\Gamma_1) \wedge v(\Gamma_2) \leq v(B) \wedge v(C) = v(B \wedge C) = v(A) \]
        \item (\( \to \)-I) In this case, \( A = B \to C \) and we have \( \Gamma \cup \qty{B} \vdash C \).
        By the inductive hypothesis, \( v\qty(\bigwedge\Gamma) \wedge v(B) \leq v(C) \).
        But then \( v\qty(\bigwedge\Gamma) \leq v(B) \Rightarrow v(C) \) by definition, so \( v\qty(\bigwedge\Gamma) \leq v(B \to C) \) as required.
        \item (\( \vee \)-I) In this case, \( A = B \vee C \), and without loss of generality, we have \( \Gamma \vdash B \).
        By the inductive hypothesis, \( v\qty(\bigwedge \Gamma) \leq v(B) \), but \( v(B) \leq v(B) \vee v(C) = v(B \vee C) \) as required.
        \item (\( \wedge \)-E) By the inductive hypothesis, we have \( v\qty(\bigwedge \Gamma) \leq v(A \wedge B) = v(A) \wedge v(B) \leq v(A), v(B) \) as required.
        \item (\( \to \)-E) We know that \( v(A \to B) = (v(A) \Rightarrow v(B)) \).
        From the inequality \( v(A \to B) \leq (v(A) \Rightarrow v(B)) \), we deduce \( v(A \to B) \wedge v(A) \leq v(B) \).
        Thus, if \( v\qty(\bigwedge \Gamma) \leq v(A \to B) \) and \( v\qty(\bigwedge \Gamma) \leq v(A) \), we have \( v\qty(\bigwedge \Gamma) \leq v(B) \) as required.
        \item (\( \vee \)-E) By the inductive hypothesis,
        \[ v\qty(A \wedge \bigwedge \Gamma) \leq v(C);\quad v\qty(B \wedge \bigwedge \Gamma) \leq v(C);\quad v\qty(\bigwedge \Gamma) \leq v(A \vee B) = v(A) \vee v(B) \]
        Hence,
        \[ v\qty(\bigwedge \Gamma) = v\qty(\bigwedge \Gamma) \wedge (v(A) \vee v(B)) = \qty(v\qty(\bigwedge \Gamma) \wedge v(A)) \vee \qty(v\qty(\bigwedge \Gamma) \wedge v(B)) \leq v(C) \vee v(C) = v(C) \]
        as every Heyting algebra is a distributive lattice.
        \item (\( \bot \)-E) If \( v\qty(\bigwedge\Gamma) \leq v(\bot) = \bot \), then \( v\qty(\bigwedge \Gamma) = \bot \).
        Hence, \( v\qty(\bigwedge \Gamma) \leq v(A) \) for any \( A \).
    \end{enumerate}
\end{proof}
\begin{example}
    The law of the excluded middle \( \mathsf{LEM} \) is not provable in \( \mathsf{IPC} \).
    Let \( p \) be a primitive proposition, and consider the Heyting algebra given by the Sierpi\'nski topology \( \qty{\varnothing, \qty{1}, \qty{1, 2}} \) on \( X = \qty{1,2} \).
    We define the valuation given by \( v(p) = \qty{1} \).
    Then
    \[ v(\neg p) = \qty{1} \Rightarrow \varnothing = (\qty{1, 2} \setminus \qty{1})^\circ = \qty{2}^\circ = \varnothing \]
    Hence,
    \[ v(p \vee \neg p) = \qty{1} \cup \varnothing = \qty{1} \neq \qty{1, 2} = \top \]
    Thus, by soundness, \( p \vee \neg p \) is not provable (from the empty context, which has valuation \( \top = \qty{1, 2} \)) in \( \mathsf{IPC} \).
\end{example}
\begin{example}
    Peirce's law \( ((p \to q) \to p) \to p \) is not intuitionistically valid.
    Let \( H \) be the Heyting algebra given by the usual topology on the plane \( \mathbb R^2 \), and let
    \[ v(p) = \mathbb R^2 \setminus \qty{(0, 0)};\quad v(q) = \varnothing \]
    % TODO: complete
\end{example}
Classical completeness can be phrased as
\[ \Gamma \vdash_{\mathsf{CPC}} A \iff \Gamma \vDash_2 A \]
where \( 2 \) is the Boolean algebra \( \qty{0, 1} \).
For intuitionistic logic, we cannot replace \( 2 \) with a single finite Heyting algebra, so we will instead quantify over all Heyting algebras.
\begin{theorem}[completeness]
    A proposition is provable in \( \mathsf{IPC} \) if and only if it is \( H \)-valid for every Heyting algebra \( H \).
\end{theorem}
\begin{proof}
    For the forward direction, if \( \vdash_{\mathsf{IPC}} A \), then \( \top \leq v(A) \) for every Heyting algebra \( H \) and valuation \( v \), by soundness.
    Then \( \top = v(A) \), so \( A \) is \( H \)-valid.

    For the backward direction, suppose \( A \) is \( H \)-valid for every Heyting algebra \( H \).
    Note that the Lindenbaum--Tarski algebra \( \faktor{\mathcal L}{\sim} \) for the empty theory, with respect to \( \mathsf{IPC} \), is a Heyting algebra.
    Consider the valuation given by mapping each primitive proposition to its equivalence class in \( \faktor{\mathcal L}{\sim} \).
    Then, one can easily show by induction that \( v : \mathcal L \to \faktor{\mathcal L}{\sim} \) is the quotient map by considering the construction of the Lindenbaum--Tarski algebra.
    Now, \( A \) is valid in every Heyting algebra and with respect to every valuation, so in particular, \( v(A) = \top \) in \( \faktor{\mathcal L}{\sim} \).
    But then \( v(A) \in [\top] \), so \( \vdash_{\mathsf{IPC}} A \leftrightarrow \top \), so \( \vdash_{\mathsf{IPC}} A \) as required.
\end{proof}

\subsection{Kripke semantics}
\begin{definition}
    Let \( S \) be a poset.
    For each \( a \in S \), we define its \emph{principal up-set} to be
    \[ {a\!\uparrow} = \qty{s \in S \mid a \leq s} \]
\end{definition}
Note that \( U \subseteq S \) is a terminal segment if and only if it contains \( a\!\uparrow \) for each \( a \in U \).
\begin{proposition}
    Let \( S \) be a poset.
    Then the set \( T(S) \) of terminal segments of \( S \) has the structure of a Heyting algebra.
\end{proposition}
\begin{proof}
    The order is given by inclusion: \( U \leq V \) if and only if \( U \subseteq V \).
    We define
    \begin{align*}
        U \wedge V &= U \cap V \\
        U \vee V &= U \cup V \\
        U \Rightarrow V &= \qty{s \mid {s\!\uparrow} \cap U \subseteq V}
    \end{align*}
    One can check that this forms a Heyting algebra as required.
\end{proof}
\begin{definition}
    Let \( P \) be a set of primitive propositions.
    A \emph{Kripke model} is a triple \( (S, \leq, \Vdash) \) where \( S \) is a poset and \( (\Vdash) \subseteq S \times P \) is a relation satisfying the \emph{persistence property}: if \( p \in P \) is such that \( s \Vdash p \) and \( s \leq s' \), then \( s' \Vdash p \).
\end{definition}
\( S \) is a set of possible \emph{worlds}, or states of knowledge, ordered by how knowledgeable they are.
The relation \( \Vdash \) is called the \emph{forcing} relation; we say that a world \emph{forces} a proposition to be true.

Every valuation \( v \) on \( T(S) \) induces a Kripke model by setting \( s \Vdash p \iff s \in v(p) \).
The persistence property corresponds to the fact that \( T(S) \) contains only terminal segments.
\begin{definition}
    Let \( (S, \leq, \Vdash) \) be a Kripke model.
    We can extend the forcing relation to a relation \( (\Vdash) \subseteq S \times \mathcal L \) recursively as follows.
    \begin{enumerate}
        \item \( s \nVdash \bot \);
        \item \( s \Vdash \varphi \wedge \psi \) if and only if \( s \Vdash \varphi \) and \( s \Vdash \psi \);
        \item \( s \Vdash \varphi \vee \psi \) if and only if \( s \Vdash \varphi \) or \( s \Vdash \psi \);
        \item \( s \Vdash \varphi \to \psi \) if and only if for all \( s' \geq s \), \( s' \Vdash \varphi \) implies \( s' \Vdash \psi \).
    \end{enumerate}
\end{definition}
One can check by induction that persistence holds for arbitrary propositions.
\begin{remark}
    \( s \Vdash \neg\varphi \) if and only if no more knowledgeable world than \( s \) forces \( \varphi \).
    \( s \Vdash \neg\neg\varphi \) is the statement that \( \varphi \) is consistent with every extension of \( s \) but need not hold in \( s \) itself; that is, for each \( s' \geq s \), there exists \( s'' \geq s' \) with \( s \Vdash \varphi \).
\end{remark}
We say that \( S \Vdash \varphi \) if every world \( s \) forces \( \varphi \).
If \( S \) has a bottom element \( s \), then \( S \Vdash \varphi \) if and only if \( s \Vdash \varphi \) by persistence.
\begin{example}
    Consider the Kripke models
    \begin{enumerate}
        \item % https://q.uiver.app/#q=WzAsMixbMCwxLCJzIl0sWzAsMCwicyciXSxbMCwxLCIiLDAseyJzdHlsZSI6eyJoZWFkIjp7Im5hbWUiOiJub25lIn19fV1d
        \[\begin{tikzcd}
            {s'} \\
            s
            \arrow[no head, from=2-1, to=1-1]
        \end{tikzcd}\]
            where \( s' \Vdash p \);
        \item % https://q.uiver.app/#q=WzAsMyxbMSwxLCJzIl0sWzAsMCwicyciXSxbMiwwLCJzJyciXSxbMCwxLCIiLDAseyJzdHlsZSI6eyJoZWFkIjp7Im5hbWUiOiJub25lIn19fV0sWzAsMiwiIiwyLHsic3R5bGUiOnsiaGVhZCI6eyJuYW1lIjoibm9uZSJ9fX1dXQ==
        \[\begin{tikzcd}
            {s'} && {s''} \\
            & s
            \arrow[no head, from=2-2, to=1-1]
            \arrow[no head, from=2-2, to=1-3]
        \end{tikzcd}\]
        where \( s'' \Vdash p \);
        \item % https://q.uiver.app/#q=WzAsMixbMCwxLCJzIl0sWzAsMCwicyciXSxbMCwxLCIiLDAseyJzdHlsZSI6eyJoZWFkIjp7Im5hbWUiOiJub25lIn19fV1d
        \[\begin{tikzcd}
            {s'} \\
            s
            \arrow[no head, from=2-1, to=1-1]
        \end{tikzcd}\]
            where \( s' \Vdash p \) and \( s' \Vdash q \).
    \end{enumerate}
    Note that in (i), we have \( s \nVdash \neg p \), since \( s' \geq s \) and \( s' \Vdash p \).
    But also \( s \nVdash p \) by assumption, thus \( s \nVdash p \vee \neg p \).
    Note that \( s \Vdash \neg\neg p \), but \( s \nVdash p \), so we also have \( s \nVdash \neg\neg p \to p \).

    In (ii), \( s \nVdash \neg\neg p \), since \( s' \geq s \) cannot access a world that forces \( p \).
    We also have \( s \nVdash \neg p \), since \( s'' \geq s' \) and \( s'' \vDash p \).
    Thus \( s \nVdash \neg\neg p \vee \neg p \).

    In (iii), \( s \nVdash (p \to q) \to (\neg p \vee q) \).
    Indeed, all worlds force \( p \to q \), and we have \( s \nVdash q \), so it suffices to check that \( s \nVdash \neg p \), but this holds as \( s' \geq s \) and \( s' \vDash p \).
\end{example}
A filter \( \mathcal F \) is called \emph{prime} if whenever \( x \vee y \in \mathcal F \), either \( x \in \mathcal F \) or \( y \in \mathcal F \).
\begin{lemma}
    Let \( H \) be a Heyting algebra and let \( v \) be an \( H \)-valuation.
    Then there is a Kripke model \( (S, \leq, \Vdash) \) such that for each proposition \( \varphi \), we have \( v \vDash_H \varphi \) if and only if \( S \Vdash \varphi \).
\end{lemma}
Thus we can convert between Kripke models and valuations on Heyting algebras.
This will allow us to prove the completeness theorem for Kripke semantics.
\begin{proof}
    Let \( S \) be the set of prime filters on \( H \) ordered by inclusion.
    We say that \( \mathcal F \Vdash p \) if and only if \( v(p) \in \mathcal F \), and prove by induction that this extends to arbitrary propositions.
    Here, we will prove the case of implications; the other connectives are easy, and primality of the filter is required for the case of disjunction.
    Let \( \mathcal F \Vdash (\psi \to \psi') \) and suppose \( v(\psi \to \psi') = v(\psi) \Rightarrow v(\psi') \notin \mathcal F \).
    Let \( \mathcal G' \) be the smallest filter containing \( \mathcal F \) and \( v(\psi) \).
    Then
    \[ \mathcal G' = \qty{b \mid \exists f \in \mathcal F.\, f \wedge v(\psi) \leq b} \]
    Note that \( v(\psi') \notin \mathcal G' \), otherwise \( f \wedge v(\psi) \leq v(\psi') \) for some \( f \in \mathcal F \), and then \( f \leq v(\psi) \Rightarrow v(\psi') \in \mathcal F \), giving a contradiction.
    In particular, \( \mathcal G' \) is a proper filter, so by Zorn's lemma there is a prime filter \( \mathcal G \) containing \( \mathcal G' \) that does not contain \( v(\psi') \).

    By the inductive hypothesis, \( \mathcal G \Vdash \psi \), and since \( \mathcal F \Vdash (\psi \to \psi') \) and \( \mathcal G' \) contains \( \mathcal G \) which contains \( \mathcal F \), we must have \( \mathcal G \Vdash \psi' \).
    Then \( v(\psi') \in \mathcal G \), which is a contradiction.
    Thus \( \mathcal F \Vdash \psi \to \psi' \) implies that \( v(\psi \to \psi') \in \mathcal F \).

    Conversely, suppose
    \[ v(\psi \to \psi') \in \mathcal F \subseteq \mathcal G \Vdash \psi \]
    By the inductive hypothesis, \( v(\psi) \in \mathcal G \), and so \( v(\psi) \Rightarrow v(\psi') \in \mathcal G \) as \( \mathcal F \subseteq \mathcal G \).
    Then \( v(\psi') \geq v(\psi) \wedge (v(\psi) \Rightarrow v(\psi')) \in \mathcal G \), so again by the inductive hypothesis, \( G \Vdash \psi' \) as required.

    It thus suffices to show that \( v \vDash_H \varphi \) if and only if \( S \Vdash \varphi \).
    If \( v \vDash_H \varphi \), then \( v(\varphi) = \top \), so \( v(\varphi) \) is contained in every filter of \( H \).
    So \( \mathcal F \Vdash \varphi \) for every prime filter \( \mathcal F \).
    Conversely, suppose \( S \Vdash \varphi \) but \( v \nvDash_H \varphi \).
    Then since \( v(\varphi) \neq \top \), there must be a proper filter \( \mathcal F \) that does not contain \( v(\varphi) \).
    We extend this as above to a prime filter \( \mathcal G \) that does not contain \( v(\varphi) \).
    Then \( \mathcal G \nVdash \varphi \), contradicting the assumption that \( S \Vdash \varphi \).
\end{proof}
\begin{theorem}[completeness]
    For every proposition \( \varphi \), we have \( \Gamma \vdash_{\mathsf{IPC}} \varphi \) if and only if for all Kripke models \( (S, \leq, \Vdash) \), if \( S \Vdash \Gamma \) then \( S \Vdash \varphi \).
\end{theorem}
\begin{proof}
    Soundness holds by induction.
    For adequacy, suppose \( \Gamma \nvdash_{\mathsf{IPC}} \varphi \).
    Then by completeness of Heyting semantics, there is a Heyting algebra \( H \) and \( H \)-valuation \( v \) such that \( v \vDash_H \Gamma \) but \( v \nvDash_H \varphi \).
    By the previous lemma, there is a Kripke model \( (S, \leq, \Vdash) \) such that \( S \Vdash \Gamma \) but \( S \nVdash \varphi \), contradicting the hypothesis.
\end{proof}
