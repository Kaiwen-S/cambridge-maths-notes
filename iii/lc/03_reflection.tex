\subsection{The Keisler extension property}
\begin{definition}
    A cardinal \( \kappa \) has the \emph{Keisler extension property}, written \( \mathsf{KEP}(\kappa) \), if there is \( \kappa \in X \supsetneq \mathrm{V}_\kappa \) transitive such that \( \mathrm{V}_\kappa \preceq X \).
\end{definition}
\begin{proposition}
    If \( \kappa \) is inaccessible and satisfies the Keisler extension property, there is an inaccessible cardinal \( \lambda < \kappa \).
\end{proposition}
\begin{proof}
    Fix \( X \) as in the Keisler extension property.
    As \( \kappa \) is inaccessible, \( X \vDash \mathsf{I}(\kappa) \) because \( \kappa \in X \) and inaccessibility is downwards absolute for transitive models.
    Also, \( \mathrm{V}_\kappa \vDash \mathsf{ZFC} \), so \( X \vDash \mathsf{ZFC} \) as it is an elementary superstructure.
    Therefore, \( X \vDash \mathsf{ZFC} + \mathsf{IC} \), so \( \mathrm{V}_\kappa \vDash \mathsf{ZFC} + \mathsf{IC} \).
    So as inaccessibility is absolute between \( \mathrm{V}_\kappa \) and \( \mathrm{V} \), there is an inaccessible \( \lambda < \kappa \).
\end{proof}
The phenomenon that properties of \( X \) occur below \( \kappa \) is called \emph{reflection}.
This argument can be improved in the following sense.
For a given \( \alpha < \kappa \),
\[ X \vDash \exists \lambda > \alpha.\, \mathsf{I}(\lambda) \]
But as \( \alpha \in \mathrm{V}_\kappa \), elementarity gives
\[ \mathrm{V}_\kappa \vDash \exists \lambda > \alpha.\, \mathsf{I}(\lambda) \]
So the set
\[ \qty{\lambda < \kappa \mid \mathsf{I}(\lambda)} \]
is not only nonempty, but cofinal in \( \kappa \).
\begin{corollary}
    Let \( \mathsf{A} \) be the axiom
    \[ \exists \kappa.\, \mathsf{I}(\kappa) \wedge \mathsf{KEP}(\kappa) \]
    Then
    \[ \mathsf{ZFC} + \mathsf{IC} <_{\Con} \mathsf{ZFC} + \mathsf{A} \]
\end{corollary}
\begin{proof}
    It suffices to show that \( \mathsf{ZFC} + \mathsf{A} \vDash \Con(\mathsf{ZFC} + \mathsf{IC}) \).
    We have seen that \( \mathsf{ZFC} + \mathsf{A} \) proves the existence of (at least) two inaccesible cardinals below \( \kappa \), and in particular the larger of the two is a model of \( \mathsf{ZFC} + \mathsf{IC} \).
\end{proof}
\begin{remark}
    This is the main technique for proving strict inequalities of consistency strength.
    Given two large cardinal properties \( \Phi, \Psi \) with the appropriate amount of absoluteness properties, we show that \( \mathsf{ZFC} + \Phi(\kappa) \) proves that the set
    \[ \qty{\lambda < \kappa \mid \Psi(\lambda)} \]
    is cofinal in \( \kappa \).
    Then \( \mathsf{ZFC} + \Phi\mathsf{C} \vDash \Con(\mathsf{ZFC} + \Psi\mathsf{C}) \).
\end{remark}
\begin{example}
    Consider the proof that every inaccessible cardinal has a worldly cardinal below it.
    In the construction, we produce a sequence of ordinals \( (\alpha_i)_{i \in \omega} \), and the worldly cardinal is \( \sup \alpha_i \).
    But we can set \( \alpha_0 = \lambda + 1 \) for a given worldly cardinal \( \lambda < \kappa \), so this gives a cofinal sequence of worldly cardinals below every given inaccessible.
\end{example}
\begin{theorem}
    Every strongly compact cardinal has the Keisler extension property.
\end{theorem}
\begin{proof}
    We want to use the method of (elementary) diagrams to produce a model with \( \mathrm{V}_\kappa \) as a substructure.
    However, we have no way to control whether such a model is well-founded using standard first-order model-theoretic techniques.
    To bypass this issue, we will use infinitary operators.

    Let \( c_x \) be a constant symbol for each \( x \in \mathrm{V}_\kappa \), and let \( L \) be the language with \( \in \) and the \( c_x \).
    Let
    \[ \mathcal V = (\mathrm{V}_\kappa, \in, \qty{x \mid x \in \mathrm{V}_\kappa}) \]
    In first-order logic, \( \mathrm{Th}(X) \) is the elementary diagram of \( \mathrm{V}_\kappa \), so if \( M \vDash \mathrm{Th}(X) \), then \( \mathrm{V}_\kappa \subseteq M \).
    Let \( L_{\kappa} \) be the \( \mathcal L_{\kappa\kappa} \)-language with the same symbols.
    Consider
    \[ \psi \equiv \forall^\omega \vb v.\, \bigvee_{i \in \omega} v_{i+1} \notin v_i \]
    This expresses well-foundedness (assuming \( \mathsf{AC} \)).
    Writing \( \Phi = \mathrm{Th}_{L_\kappa}(\mathcal V) \) for the \( L_\kappa \)-theory of \( \mathcal V \), we must have \( \psi \in \Phi \) since \( \mathrm{V}_\kappa \) is well-founded.
    Thus, if \( M \vDash \Phi \), then \( M \) is a well-founded model containing \( \mathrm{V}_\kappa \).
    By taking the Mostowski collapse, we may also assume that any such \( M \) is transitive.

    Extend \( L_\kappa \) to \( L_\kappa^+ \) with one extra constant \( c \), and let
    \[ \Phi^+ = \Phi \cup \qty{c \text{ is an ordinal}} \cup \qty{c \neq c_x \mid x \in \mathrm{V}_\kappa} \]
    Any model of \( \Phi^+ \) induces a transitive elementary superstructure of \( \mathrm{V}_\kappa \) that contains an ordinal at least \( \kappa \), so by transitivity, \( \kappa \) is in this model.

    We show that \( \Phi^+ \) is satisfiable by showing that it is \( \kappa \)-satisfiable, using the fact that \( \kappa \) is strongly compact.
    Let \( \Phi^0 \subseteq \Phi^+ \) be a subset of size less than \( \kappa \).
    Then we can interpret \( c \) as some ordinal \( \alpha \) greater than all ordinals \( \beta \) occurring in the sentences \( c \neq c_\beta \) in \( \Phi^+ \).
    Then \( \mathcal V \), together with this interpretation of \( c \), is a model of \( \Phi_0 \).
\end{proof}
\begin{corollary}
    \[ \mathsf{ZFC} + \mathsf{IC} <_{\Con} \mathsf{ZFC} + \mathsf{SCC} \]
\end{corollary}
The proof above only used languages with at most \( \kappa \)-many symbols.
Let \( \mathsf{WC}(\kappa) \) be the axiom that every \( \mathcal L_{\kappa\kappa} \)-language with at most \( \kappa \)-many symbols satisfies \( \kappa \)-compactness.
Then we have shown that \( \mathsf{WC}(\kappa) \) implies the Keisler extension property.
One can show that
\[ \mathsf{W}(\kappa) \leftrightarrow \mathsf{WC}(\kappa) \]
So the cardinals \( \kappa \) that satisfy \( \mathsf{WC}(\kappa) \) are precisely the weakly compact cardinals.
In particular,
\[ \mathsf{ZFC} + \mathsf{IC} <_{\Con} \mathsf{ZFC} + \mathsf{WCC} \]
Note that in the proof that strongly compact cardinals are measurable, we used a language with \( 2^\kappa \)-many symbols.

\subsection{Ultrapowers of the universe}
In order to avoid proper classes, we will consider ultrapowers of particular set universes.
Later, we will briefly explain how all of this could have been done in a proper class universe such as \( \mathrm{V} \).
For convenience, we will assume that \( \kappa < \lambda \) where \( \kappa \) is measurable and \( \lambda \) is inaccessible, so \( \mathrm{V}_\lambda \vDash \mathsf{ZFC} + \mathsf{MC} \).
We will take the ultrapower of \( \mathrm{V}_\lambda \).

Let \( U \) be a \( \kappa \)-complete nonprincipal ultrafilter on \( \kappa \), and form the ultrapower of \( \mathrm{V}_\lambda \), consisting of equivalence classes of functions \( f : \kappa \to \mathrm{V}_\lambda \) where \( f \sim g \) when \( \qty{\alpha \mid f(\alpha) = g(\alpha)} \in U \).
\[ \faktor{{\mathrm{V}_\lambda}^\kappa}{U} = \qty{[f] \mid f : \kappa \to \mathrm{V}_\lambda} \]
The membership relation on the ultrapower is given by
\[ [f] \mathrel{E} [g] \leftrightarrow \qty{\alpha \mid f(\alpha) \in g(\alpha)} \in U \]
We have an embedding \( \ell \) from \( \mathrm{V}_\lambda \) into the ultrapower by mapping \( x \in \mathrm{V}_\lambda \) to the equivalence class of its constant function \( c_x : \kappa \to \mathrm{V}_\lambda \).
This is an elementary embedding by \L{}o\'s' theorem.
Hence
\[ \qty(\mathrm{V}_\lambda, \in) \equiv \qty(\faktor{{\mathrm{V}_\lambda}^\kappa}{U}) \]
so they both model \( \mathsf{ZFC} + \mathsf{MC} \), and in particular, \( [c_{\kappa}] \) is a measurable cardinal.
\begin{remark}
    \begin{enumerate}
        \item Suppose \( \faktor{{\mathrm{V}_\lambda}^\kappa}{U} \vDash [f] \text{ is an ordinal} \).
        By \L{}o\'s' theorem,
        \[ X = \qty{\alpha \mid f(\alpha) \text{ is an ordinal}} \in U \]
        We can define
        \[ f'(\alpha) = \begin{cases}
            f(\alpha) & \text{if } \alpha \in X \\
            0 & \text{otherwise}
        \end{cases} \]
        Note that \( f \sim f' \), so \( [f] = [f'] \).
        So without loss of generality, we can assume \( f \) is a function into \( \mathrm{Ord} \cap \lambda = \lambda \), so \( f : \kappa \to \lambda \).
        Since \( \lambda \) is inaccessible, \( f \) cannot be cofinal, so there is \( \gamma < \lambda \) such that \( f : \kappa \to \gamma \).
        Note also that, for example, we can define \( f + 1 \) by
        \[ (f + 1)(\alpha) = f(\alpha) + 1 \]
        so
        \[ \qty{\alpha \mid (f+1)(\alpha) \text{ is the successor of } f(\alpha)} = \kappa \in U \]
        hence by \L{}o\'s' theorem, \( [f + 1] \) is the successor of \( [f] \).
        \item If \( f : \kappa \to \mathrm{V}_\lambda \) is arbitrary, the set
        \[ \qty{\rank f(\alpha) \mid \alpha \in \kappa} \]
        cannot be cofinal in \( \lambda \), so there is \( \gamma < \lambda \) such that \( f \in \mathrm{V}_\gamma \).
        However, the union of the equivalence class \( [f] \) is unbounded in \( \mathrm{V}_\lambda \).
        \item Given \( f \), by (ii) we may assume \( f \in \mathrm{V}_\gamma \) for some \( \gamma < \lambda \).
        If \( [g] \mathrel{E} [f] \), then
        \[ X = \qty{\alpha \mid g(\alpha) \in f(\alpha)} \in U \]
        Now we can define
        \[ g'(\alpha) = \begin{cases}
            g(\alpha) & \text{if } \alpha \in X \\
            0 & \text{otherwise}
        \end{cases} \]
        Then \( g \sim g' \) so \( [g] = [g'] \), and \( g' \in \mathrm{V}_\gamma \).
        Therefore,
        \[ \abs{\qty{[g] \mid [g] \mathrel{E} [f]}} \leq \abs{\mathrm{V}_\gamma} < \lambda \]
    \end{enumerate}
\end{remark}
\begin{lemma}
    \( \faktor{{\mathrm{V}_\lambda}^\kappa}{U} \) is \( E \)-well-founded.
\end{lemma}
\begin{proof}
    Suppose not, so let \( \qty{[f_n] \mid n \in \mathbb N} \) be a strictly decreasing sequence, so
    \[ [f_{n+1}] \mathrel{E} [f_n] \]
    By definition,
    \[ X_n = \qty{\alpha \mid f_{n+1}(\alpha) \in f_n(\alpha)} \in U \]
    But as \( U \) is \( \kappa \)-complete,
    \[ \bigcap_{n \in \mathbb N} X_n \in U \]
    In particular, there must be an element \( \alpha \in \bigcap_{n \in \mathbb N} X_n \).
    Hence, \( f_n(\alpha) \) is an \( \in \)-decreasing sequence in \( \mathrm{V}_\lambda \), which is a contradiction.
\end{proof}
Note that we only used \( \aleph_1 \)-completeness of \( U \).

We can take the Mostowski collapse to produce a transitive set \( M \) such that
\[ \pi : \qty(\faktor{{\mathrm{V}_\lambda}^\kappa}{U}, E) \cong (M, \in) \]
Combining \( \ell \) and \( \pi \), we obtain
\[ j = \pi \circ \ell : (\mathrm{V}_\lambda, \in) \to (M, \in) \]
given by
\[ j(x) = \pi(\ell(x)) = \pi([c_x]) \]
For convenience, will write \( (f) \) to abbreviate \( \pi([f]) \), so \( j(x) = (c_x) \).
\begin{lemma}
    \( M \subseteq \mathrm{V}_\lambda \).
\end{lemma}
\begin{proof}
    Note that because \( \lambda \) is inaccessible, \( \mathrm{V}_\lambda = \mathrm{H}_\lambda \), where
    \[ \mathrm{H}_\lambda = \qty{x \mid \abs{\operatorname{tcl}(x)} < \lambda} \]
    Since \( M \) is transitive, if \( \abs{x} < \lambda \) for each \( x \in M \), then \( M \subseteq \mathrm{H}_\lambda \).
    But remark (iii) above shows precisely what is required.
\end{proof}
\begin{lemma}
    \( \mathrm{Ord} \cap M = \lambda \).
\end{lemma}
\begin{proof}
    Under the elementary embedding \( j \), ordinals in \( \mathrm{V}_\lambda \) are mapped to ordinals in \( M \).
    So \( j \) restricts to an order-preserving embedding from \( \lambda \) into a subset of \( \lambda \).
    Thus this embedding is unbounded, and therefore by transitivity, \( \mathrm{Ord} \cap M = \lambda \).
\end{proof}
\begin{lemma}
    \( \eval{j}_{\mathrm{V}_\kappa} = \id \), so in particular, \( \mathrm{V}_\kappa \subseteq M \).
\end{lemma}
\begin{proof}
    We show this by \( \in \)-induction on \( \mathrm{V}_\kappa \).
    Suppose that \( x \in \mathrm{V}_\kappa \) is such that for all \( y \in x \), \( j(y) = y \).
    For any \( y \in x \), by elementarity, \( j(y) \in j(x) \), but \( j(y) = y \) so \( y \in j(x) \) as required.
    For the converse, suppose \( y \in j(x) \).
    Then define \( f \) such that \( y = (f) \), so \( (f) \in (c_x) \).
    Hence
    \[ X = \qty{\alpha \mid f(\alpha) \in c_x(\alpha)} = \qty{\alpha \mid f(\alpha) \in x} \in U \]
    But
    \[ \qty{\alpha \mid f(\alpha) \in x} = \bigcup_{z \in x} \qty{\alpha \mid f(\alpha) = z} \]
    This is a union of \( \abs{x} \)-many sets.
    By \( \kappa \)-completeness, there must be some \( z \in x \) such that
    \[ \qty{\alpha \mid f(\alpha) = z} \in U \]
    Hence \( f \sim c_z \).
    Therefore, \( (f) = j(z) \), and by the inductive hypothesis, \( j(z) = z \).
    Hence \( y \in x \).
\end{proof}
\begin{lemma}
    \( j \neq \id \), as \( j(\kappa) > \kappa \).
\end{lemma}
\begin{proof}
    We know that \( j(\kappa) = (c_\kappa) \).
    By the previous lemma, for each \( \alpha < \kappa \), \( j(\alpha) = (c_\alpha) = \alpha \).
    Consider the identity map \( \id_\kappa : \kappa \to \kappa \).
    We have
    \begin{align*}
        (c_\alpha) < (\id_\kappa) &\leftrightarrow \qty{\gamma \mid c_\alpha(\gamma) < \id_\kappa(\gamma)} \in U \\
        &\leftrightarrow \qty{\gamma \mid \alpha < \gamma} \in U
    \end{align*}
    But by a size argument, \( \qty{\gamma \mid \gamma \leq \alpha} \notin U \) as \( U \) is nonprincipal, so we must have \( \alpha < (\id) \).
    Also,
    \begin{align*}
        (\id_\kappa) < (c_\kappa) &\leftrightarrow \qty{\gamma \mid \id_\kappa(\gamma) < c_\kappa(\gamma)} \in U \\
        &\leftrightarrow \qty{\gamma \mid \gamma < \kappa} \in U
    \end{align*}
    This is certainly in \( U \).
    So for all \( \alpha < \kappa \),
    \[ \alpha < (\id_\kappa) < j(\kappa) \]
    giving
    \[ \kappa \leq (\id_\kappa) < j(\kappa) \]
    as required.
\end{proof}
\begin{remark}
    \begin{enumerate}
        \item This implies that \( \eval{j}_{\mathrm{V}_{\kappa + 1}} \neq \id \), so the identity result above cannot be strengthened.
        \item This also shows that many of the elements of \( M \) arise from non-constant functions.
        \item The set
        \[ \qty{j(x) \mid x \in \mathrm{V}_\lambda} \]
        is isomorphic to \( \mathrm{V}_\lambda \).
        Therefore, there is a (non-transitive) copy of \( \mathrm{V}_\lambda \) that sits strictly inside \( M \).
        \item Let \( f : \kappa \to \kappa \) be a function such that for all \( \gamma < \kappa \), \( \id_\kappa(\gamma) < f(\gamma) \).
        Then \( (\id_\kappa) < (f) \).
        For example, the functions \( f_2(\gamma) = \gamma \cdot 2 \) and \( f_3(\gamma) = \gamma \cdot 3 \) satisfy \( (\id_\kappa) < (f_2) < (f_3) \).
        \item At the moment, we do not know whether \( (\id_\kappa) = \kappa \).
        Consider
        \[ f(\gamma) = \begin{cases}
            \gamma - 1 & \text{if } \gamma \text{ is a successor} \\
            \gamma & \text{if } \gamma \text{ is a limit}
        \end{cases} \]
        Then
        \[ (f) < (\id_\kappa) \leftrightarrow \qty{\alpha \mid \alpha \text{ is a limit}} \notin U \]
        We will discuss this in more detail later.
    \end{enumerate}
\end{remark}

\subsection{Properties above the critical point}
\begin{definition}
    Let \( j : \mathrm{V}_\lambda \to M \) be an elementary embedding such that \( M \subseteq \mathrm{V}_\lambda \) is transitive.
    An ordinal \( \mu \) is called the \emph{critical point} of \( j \), written \( \operatorname{crit}(j) \), if \( j \neq \id \) and \( \mu \) is the least ordinal \( \alpha \). such that \( j(\alpha) > \alpha \).
\end{definition}
Note that if \( j \neq \id \), it moves the rank of some set, so moves some ordinal.
Therefore, if \( j \neq \id \), it has a critical point.

In this terminology, the critical point of the embedding \( j \) above is \( \kappa \).
\begin{remark}
    \begin{enumerate}
        \item \( M \) is closed under finite intersections: if \( A, B \in M \), then \( A \cap B \in M \).
        \item \( \mathrm{V}_\kappa \in M \).
        To show this, we claim that the set
        \[ W = \qty{y \in M \mid M \vDash \rank y < \kappa} \]
        is equal to \( \mathrm{V}_\kappa \).
        Then, since \( M \) models \( \mathsf{ZFC} \), the set \( W \) is \( \mathrm{V}_\kappa^M \), so \( W \in M \).

        If \( x \in \mathrm{V}_\kappa \), then \( \rank x = \alpha < \kappa \), so \( j(x) = x \).
        By elementarity, \( \rank x = \rank j(x) = j(\alpha) = \alpha \) as required.
        Conversely, suppose that \( M \vDash \rank y = \gamma \) for \( \gamma < \kappa \).
        There is \( f \) such that \( y = (f) \), and without loss of generality we can take \( f : \kappa \to \mathrm{V}_{\gamma + 1} \).
        But \( \abs{\mathrm{V}_{\gamma + 1}} < \kappa \), and so by the argument in the lemma proving \( \eval{j}_{\mathrm{V}_\kappa} = \id \), there is some \( x \in \mathrm{V}_{\gamma + 1} \) such that \( \qty{\alpha \mid f(\alpha) = x} \in U \).
        Hence \( f \sim c_x \), and so \( y = j(x) = x \).
    \end{enumerate}
\end{remark}
\begin{lemma}
    \( \mathrm{V}_{\kappa + 1} \subseteq M \).
\end{lemma}
Note that \( \eval{j}_{\mathrm{V}_{\kappa + 1}} \neq \id \).
\begin{proof}
    Let \( A \in \mathrm{V}_{\kappa + 1} \), so \( A \subseteq \mathrm{V}_\kappa \).
    We claim that \( A = j(A) \cap \mathrm{V}_\kappa \).
    Then, by the two remarks above, this implies \( A \in M \).

    Suppose \( x \in A \subseteq \mathrm{V}_\kappa \).
    By elementarity, \( j(x) \in j(A) \), but \( x = j(x) \), so \( x \in j(A) \).
    Conversely, suppose \( x \in j(A) \cap \mathrm{V}_\kappa \).
    Then \( x = j(x) \), so \( j(x) \in j(A) \).
    So by elementarity in the other direction, \( x \in A \).
\end{proof}
\begin{lemma}
    \( \mathrm{V}_\lambda \vDash \abs{j(\kappa)} \leq 2^\kappa \).
\end{lemma}
\begin{proof}
    Recall that if \( f \in \mathrm{V}_\gamma \) then \( \abs{(f)} \leq \abs{\mathrm{V}_\gamma} \).
    So if \( (f) \in j(\kappa) = (c_\kappa) \), we can assume \( f : \kappa \to \kappa \), and there are only \( 2^\kappa \)-many such functions.
\end{proof}
In particular, \( \mathrm{V}_\lambda \) believes that \( j(\kappa) \) is not a strong limit cardinal.
Hence,
\begin{lemma}
    \( M \neq \mathrm{V}_\lambda \).
\end{lemma}
\begin{proof}
    \( M \) believes that \( j(\kappa) \) is measurable, so in particular it believes \( j(\kappa) \) is a strong limit.
    Hence \( M \neq \mathrm{V}_\lambda \).
\end{proof}
There is a strengthening of this result which exhibits a witness to \( M \subsetneq \mathrm{V}_\lambda \), discussed on the example sheets.
Namely, we can show that \( U \notin M \).
In order to show this, we prove that for arbitrary transitive \( N \subseteq \mathrm{V}_\lambda \) with \( U \in N \), we have \( N \vDash \abs{j(\kappa)} \leq 2^\kappa \).
In particular, \( \mathrm{V}_{\kappa + 2} \nsubseteq M \).
% TODO: the drawing(s) from the lecture on 23rd of February

Note that \( M \) might still believe that \( \kappa \) is measurable, even though \( U \notin M \).
There could be some other \( U' \in V_{\kappa + 2} \) which is \( \kappa \)-complete and nonprincipal.

Recall that the Keisler extension property for a transitive model \( X \) is the statement that there is \( \kappa \in X \) such that \( \mathrm{V}_\kappa \preceq X \).
Properties of \( X \) reflect down into \( \mathrm{V}_\kappa \): if \( \alpha \in \mathrm{Ord}^{\mathrm{V}_\kappa} \) and \( \Phi \) is a property such that \( X \vDash \Phi(\kappa) \), then
\[ X \vDash \exists \mu.\, \alpha < \mu \wedge \Phi(\mu) \]
so
\[ \mathrm{V}_\kappa \vDash \exists \mu.\, \alpha < \mu \wedge \Phi(\mu) \]
hence
\[ C_\Phi = \qty{\gamma < \kappa \mid \Phi(\gamma)} \subseteq \kappa \]
is cofinal in \( \kappa \).
Now, if \( \Phi \) is any property such that \( M \vDash \Phi(\kappa) \), then for any \( \alpha < \kappa \),
\[ M \vDash \exists \mu.\, j(\alpha) < \mu < j(\kappa) \wedge \Phi(\mu) \]
By elementarity,
\[ \mathrm{V}_\lambda \vDash \exists \mu.\, \alpha < \mu < \kappa \wedge \Phi(\mu) \]
Note that \( \alpha = j(\alpha) \).
So
\[ C_\Phi = \qty{\gamma < \kappa \mid \Phi(\gamma)} \]
is cofinal in \( \kappa \).
\begin{example}
    \begin{enumerate}
        \item Let \( \Phi(\kappa) = \mathsf{I}(\kappa) \) be the statement that \( \kappa \) is inaccessible.
        By absoluteness, \( M \vDash \mathsf{I}(\kappa) \), so
        \[ C_{\mathsf{I}} = \qty{\gamma < \kappa \mid \mathsf{I}(\gamma)} \]
        is cofinal.
        So if \( \kappa \) is measurable, it is the \( \kappa \)th inaccessible cardinal.
        \item Let \( \Phi(\kappa) = \mathsf{W}(\kappa) \) be the statement that \( \kappa \) is weakly compact.
        We show that \( M \vDash \mathsf{W}(\kappa) \).
        Let \( c : [\kappa]^2 \to 2 \) be a colouring in \( M \); we find \( H \in [\kappa]^\kappa \) in \( M \) that is monochromatic for \( c \).
        By the fact that \( \mathrm{V}_\lambda \vDash \mathsf{W}(\kappa) \), we obtain \( H \) as above in \( \mathrm{V}_\lambda \).
        But this \( H \) is a subset of \( \kappa \), so is an element of \( \mathrm{V}_{\kappa + 1} \subseteq M \) as required.
        By the reflection argument,
        \[ C_{\mathsf{W}} = \qty{\gamma < \kappa \mid \mathsf{W}(\gamma)} \]
        is cofinal in \( \kappa \).
        So the least weakly compact cardinal is not measurable.
    \end{enumerate}
\end{example}
\begin{definition}
    A property \( \Phi \) is called \emph{\( \beta \)-stable} if for all transitive models \( M \) and all \( \kappa \), if \( \Phi(\kappa) \) holds and \( \mathrm{V}_{\kappa + \beta} \subseteq M \) then \( M \vDash \Phi(\kappa) \).
\end{definition}
\begin{remark}
    \begin{enumerate}
        \item Weak compactness is 1-stable, and 1-stable properties of measurable cardinals reflect at a measurable cardinal.
        \item Measurability is 2-stable, because the property \( \Xi \) of being a \( \kappa \)-complete nonprincipal ultrafilter is absolute, but the existence of the ultrafilter requires two power set operations:
        \[ \mathsf{M}(\kappa) \leftrightarrow \exists U \in \mathrm{V}_{\kappa + 2}.\, \Xi(U) \]
    \end{enumerate}
\end{remark}
\begin{example}
    Suppose that \( M \vDash \mathsf{M}(\kappa) \).
    Then by the same reflection argument, the set \( C_{\mathsf{M}} \) is cofinal in \( \kappa \), so \( \kappa \) is the \( \kappa \)th measurable cardinal, and so is not the least.
\end{example}
\begin{definition}
    A cardinal \( \kappa \) is called \emph{surviving}, written \( \mathsf{Surv}(\kappa) \), if there is \( \lambda > \kappa \) inaccessible, a \( \kappa \)-complete nonprincipal ultrafilter on \( \kappa \), a transitive model \( M \) such that \( M \cong \faktor{{\mathrm{V}_\lambda}^\kappa}{U} \) and \( j \) is the elementary embedding derived from \( U \), where \( M \vDash \mathsf{M}(\kappa) \).
\end{definition}
By the example above, if \( \kappa \) is the first surviving cardinal, it is the \( \kappa \)th measurable.
Under sufficient consistency assumptions, we have the following.
\begin{corollary}
    \( \mathsf{MC} <_{\Con} \mathsf{SurvC} \).
\end{corollary}
\begin{proof}
    Let \( \kappa \) be a surviving cardinal.
    By the previous results, we can find \( \lambda_0 < \lambda_1 < \kappa \) such that \( \lambda_0, \lambda_1 \) are both measurable.
    Then \( \lambda_1 \) is inaccessible, so \( \mathrm{V}_{\lambda_1} \vDash \mathsf{ZFC} + \mathsf{M}(\lambda_0) \) by 2-stability of measurability and the fact that \( \mathrm{V}_{\lambda_0 + 2} \subseteq \mathrm{V}_{\lambda_1} \).
\end{proof}

\subsection{The fundamental theorem on measurable cardinals}
\begin{theorem}
    Suppose \( \lambda \) is inaccessible and \( \kappa < \lambda \).
    Then the following are equivalent.
    \begin{enumerate}
        \item \( \kappa \) is measurable.
        \item There is a transitive model \( M \) of \( \mathsf{ZFC} \) with \( \mathrm{V}_{\kappa + 1} \subseteq M \) and an elementary embedding \( j : \mathrm{V}_\lambda \to M \) such that \( j \neq \id \) and \( \kappa = \operatorname{crit}(j) \).
    \end{enumerate}
\end{theorem}
\begin{proof}
    We have already shown that (i) implies (ii).
    For the converse, we define an ultrafilter \( U \) by
    \[ U = \qty{A \subseteq \kappa \mid \kappa \in j(A)} \]
    Note that if \( A \subseteq \kappa \), then \( j(A) \subseteq j(\kappa) \), so it could in fact be the case that \( \kappa \in j(A) \).
    We show that \( U \) is a \( \kappa \)-complete nonprincipal ultrafilter.
    \begin{itemize}
        \item We have \( \kappa \in U \) precisely if \( \kappa \in j(\kappa) \), but this is true as \( \kappa \) is the critical point of \( j \).
        \item \( \varnothing \in U \) precisely if \( \kappa \in j(\varnothing) \), but \( j(\varnothing) = \varnothing \) as \( j \) is an elementary embedding.
        \item If \( A \in U \) and \( B \supseteq A \), then \( \kappa \in j(A) \), but \( j(B) \supseteq j(A) \) by elementarity, so \( \kappa \in j(B) \) giving \( B \in U \).
        \item Suppose \( A \notin U \).
        Then \( \kappa \notin j(A) \).
        We want to show \( \kappa \setminus A \in U \), or equivalently, \( \kappa \in j(\kappa \setminus A) \).
        By elementarity, \( j(\kappa \setminus A) = j(\kappa) \setminus j(A) \).
        But \( \kappa \in j(\kappa) \setminus j(A) \) as required.
        \item We show \( U \) is nonprincipal.
        Let \( \alpha \in \kappa \).
        Then \( \qty{\alpha} \in U \) precisely when \( \kappa \in j(\qty{\alpha}) = \qty{j(\alpha)} \).
        But \( \alpha < \kappa \), so \( j(\alpha) = \alpha \neq \kappa \), hence \( U \) cannot be principal.
        \item Finally, we show \( \kappa \)-completeness; this will also show the finite intersection property required for \( U \) to be a filter.
        Let \( \gamma < \kappa \), and fix \( (A_\alpha)_{\alpha < \gamma} \) such that \( A_\alpha \in U \) for each \( \alpha < \gamma \).
        Then \( \kappa \in j(A_\alpha) \) for all \( \alpha < \gamma \).
        Then \( \bigcap_{\alpha < \gamma} A_\alpha \in U \) if and only if \( \kappa \in j\qty(\bigcap_{\alpha < \gamma} A_\alpha) \).
        Note that being an element of \( \bigcap_{\alpha < \gamma} A_\gamma \) is a formula that says that \( \vb A \) is a sequence of objects \( A_\alpha \), the \( \alpha \)th element of this sequqence is \( A_\alpha \), and \( \beta \) is an element of each element of the sequence.
        Therefore \( \beta \in j\qty(\bigcap_{\alpha < \gamma} A_\alpha) \) if and only if \( \beta \) is an element of all elements of the sequence \( j(\vb A) \).
        Clearly, \( j(\vb A) \) is a sequence of subsets of \( j(\kappa) \) of length \( j(\gamma) = \gamma \).
        Since \( A_\alpha \) is the \( \alpha \)th element of \( \vb A \), \( j(A_\alpha) \) is the \( j(\alpha) \)th element of \( j(\vb A) \), but \( j(\alpha) = \alpha \).
        Hence \( j(\vb A) \) is the sequence \( (j(A_\alpha))_{\alpha < \gamma} \).
        Then
        \[ j\qty(\bigcap_{\alpha < \gamma} A_{\alpha}) = \bigcap_{\alpha < \gamma} j(A_\alpha) \]
        giving \( \kappa \)-completeness as required.
    \end{itemize}
\end{proof}
\begin{remark}
    Given a sequence \( \vb A \) of subsets of \( \kappa \) of length \( \gamma \), then \( j(\vb A) \) is a sequence of subsets of \( j(\kappa) \) of length \( j(\gamma) \).
    Moreover, if \( A_\alpha \) is the \( \alpha \)th element of \( \vb A \), then \( j(A_\alpha) \) is the \( j(\alpha) \)th element of \( j(\vb A) \).
    In the situation above, \( \gamma < \kappa \), so \( j(\gamma) = \gamma \) and \( j(\alpha) = \alpha \), so \( j(\vb A) = \qty{j(A_\alpha) \mid \alpha < \gamma} \).
    If, for example, \( \gamma = \kappa \), then \( j(\vb A) \) is a sequence of length \( j(\kappa) \), which is strictly longer.
    Despite this, the first \( \kappa \)-many elements of the sequence are still \( j(A_\alpha) \) for \( \alpha < \kappa \).
    Beyond \( \kappa \), we do not know what the elements of \( j(\vb A) \) look like.
    This remark suffices for the following result.
\end{remark}
\begin{proposition}
    For arbitrary embeddings \( j \) with critical point \( \kappa \), the ultrafilter \( U_j \) constructed above is normal.
\end{proposition}
\begin{proof}
    Suppose \( A_\alpha \in U_j \) for each \( \alpha < \kappa \), or equivalently, \( \kappa \in j(A_\alpha) \).
    We must show \( \kappa \in j\qty(\operatorname*{\scalerel*{\mupDelta}{\textstyle\sum}}_{\alpha < \kappa}(A_\alpha)) \).
    We have
    \begin{align*}
        \xi \in \operatorname*{\scalerel*{\mupDelta}{\textstyle\sum}}_{\alpha < \kappa}(A_\alpha) &\leftrightarrow \xi \in \bigcap_{\alpha < \xi} A_\alpha \\
        &\leftrightarrow \forall \alpha < \xi.\, \xi \in A_\alpha \\
        \xi \in j\qty(\operatorname*{\scalerel*{\mupDelta}{\textstyle\sum}}_{\alpha < \kappa}(A_\alpha)) &\leftrightarrow \forall \alpha < \xi.\, \xi \in j(\vb A)_{j(\alpha)}
    \end{align*}
    Substitute \( \kappa \) for \( \xi \) and obtain
    \begin{align*}
        \kappa \in j\qty(\operatorname*{\scalerel*{\mupDelta}{\textstyle\sum}}_{\alpha < \kappa}(A_\alpha)) &\leftrightarrow \forall \alpha < \kappa.\, \kappa \in j(\vb A)_{j(\alpha)} \\
        &\leftrightarrow \forall \alpha < \kappa.\, \kappa \in j(\vb A)_\alpha \\
        &\leftrightarrow \forall \alpha < \kappa.\, \kappa \in j(A_\alpha)
    \end{align*}
    which holds by assumption.
\end{proof}
\begin{remark}
    \begin{enumerate}
        \item This gives an alternative proof of the existence of a normal ultrafilter on a measurable cardinal.
        \item The operations \( U \mapsto j_U \) and \( j \mapsto U_j \) are not inverses in general.
        In particular, if \( U \) is not normal, \( U_{j_U} \neq U \).
    \end{enumerate}
\end{remark}
\begin{proposition}
    Let \( U \) be a \( \kappa \)-complete nonprincipal ultrafilter on \( \kappa \).
    Then the following are equivalent.
    \begin{enumerate}
        \item \( U \) is normal;
        \item \( (\id) = \kappa \).
    \end{enumerate}
\end{proposition}
This proposition provides an alternative view of reflection.
Suppose that the ultrafilter \( U \) on \( \kappa \) is normal.
If \( M \vDash \Phi(\kappa) \), then \( M \vDash \Phi((\id)) \).
By \L{}o\'s' theorem,
\[ \qty{\alpha < \kappa \mid \Phi(\id(\alpha))} \in U \]
So \( \Phi \) reflects not only on a set of size \( \kappa \), but on an ultrafilter set.
In particular, if \( \Phi = \mathsf{M} \) and \( M \vDash \mathsf{M}(\kappa) \), so if \( \kappa \) is surviving, then the set of \( \alpha \) that are measurable is in \( U \).
Using this result, we can characterise the surviving cardinals in a more elegant way.
\begin{theorem}
    \( \kappa \) is surviving if and only if there is a normal ultrafilter on \( \kappa \) such that \( \qty{\alpha < \kappa \mid \mathsf{M}(\alpha)} \in U \).
\end{theorem}
\begin{proof}
    We have just shown one direction.
    For the converse, suppose the set \( C = \qty{\alpha < \kappa \mid \mathsf{M}(\alpha)} \) is in \( U \).
    Then for each \( \alpha \in C \), one can find an \( \alpha \)-complete nonprincipal ultrafilter on \( \alpha \) called \( U_\alpha \).
    Define
    \[ f(\alpha) = \begin{cases}
        U_\alpha & \text{if } \alpha \in C \\
        \varnothing & \text{if } \alpha \notin C
    \end{cases} \]
    Thus the set of \( \alpha \) such that \( f(\alpha) \) is an \( \alpha \)-complete nonprincipal ultrafilter on \( \alpha \) is \( C \), so in \( U \).
    Equivalently, the set of \( \alpha \) such that \( f(\alpha) \) is an \( \id(\alpha) \)-complete nonprincipal ultrafilter on \( \id(\alpha) \) is in \( U \).
    So by \L{}o\'s' theorem, \( M \) believes that \( (f) \) is an \( (\id) \)-complete nonprincipal ultrafilter on \( (\id) \).
    So \( (f) \) witnesses that \( \kappa \) is measurable in \( M \).
\end{proof}
This shows that whether a cardinal \( \kappa \) is surviving depends only on \( \mathrm{V}_{\kappa + 2} \), and is therefore a 2-stable property.
\begin{definition}
    If \( U, U' \) are normal ultrafilters on \( \kappa \), we write \( U <_M U' \) if
    \[ C = \qty{\alpha \mid \mathsf{M}(\alpha)} \in U \]
    and there is a sequence of ultrafilters \( U_\alpha \) on \( \alpha \in C \) such that
    \[ A \in U' \leftrightarrow \qty{\alpha \mid A \cap \alpha \in U_\alpha} \in U \]
    This is known as the \emph{Mitchell order}.
\end{definition}
Then \( \kappa \) is surviving if and only if there are \( U, U' \) on \( \kappa \) such that \( U <_M U' \), because of the fact that if \( h(\alpha) = A \cap \alpha \) then \( (h) = A \).
Note that talking about sequences of Mitchell-ordered ultrafilters is also 2-stable.
