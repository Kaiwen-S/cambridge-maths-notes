\subsection{Strong cardinals}
\begin{definition}
    A large cardinal axiom \( \Phi\mathsf{C} \) is called an \emph{embedding axiom} if \( \Phi(\kappa) \) holds if and only if there is a transitive model \( M \) and elementary embedding \( j : \mathrm{V}_\lambda \to M \) with critical point \( \kappa \) with certain additional properties.
\end{definition}
\( \mathsf{M}(\kappa) \) is the simplest embedding axiom.
The remaining large cardinal axioms in this course will take the form of embedding axioms.
\begin{definition}
    An embedding \( j : \mathrm{V}_\lambda \to M \) with critical point \( \kappa \) is called \emph{\( \beta \)-strong} if \( \mathrm{V}_{\kappa + \beta} \subseteq M \).
    A cardinal \( \kappa \) is called \emph{\( \beta \)-strong} if there is a \( \beta \)-strong embedding with critical point \( \kappa \).
\end{definition}
\( \beta \)-stable properties are preserved by \( \beta \)-strong embeddings.
In particular, by the reflection argument, if \( \Phi \) is \( \beta \)-stable and \( \kappa \) is \( \beta \)-strong with \( \Phi(\kappa) \), then \( \kappa \) is the \( \kappa \)th cardinal with property \( \Phi \).

Note that \( \kappa \) is measurable if and only if \( \kappa \) is 1-strong, and if \( \kappa \) is 2-strong then \( \qty{\alpha < \kappa \mid \mathsf{M}(\alpha)} \) and \( \qty{\alpha < \kappa \mid \mathsf{Surv}(\alpha)} \) are of size \( \kappa \).
If we write \( \beta\mathsf{-S}(\kappa) \) to denote that \( \kappa \) is \( \beta \)-strong, then
\[ \mathsf{SurvC} <_{\Con} 2\mathsf{-S}(\kappa) \]
This also gives an example of \( j_{U_j} \neq j \), as the ultrapower embedding of any ultrafilter is never 2-strong.
\begin{definition}
    A large cardinal property \( \Phi \) is said to have \emph{witness objects} of rank \( \beta \) if there is a formula \( \Psi \) that is downwards absolute for transitive models such that
    \[ \Phi(\kappa) \leftrightarrow \forall x.\, \exists y \in \mathrm{V}_{\kappa + \beta}.\, \Psi(x, y, \kappa) \]
\end{definition}
Any large cardinal property with witness objects of rank \( \beta \) is \( \beta \)-stable.
\begin{example}
    \begin{enumerate}
        \item Weakly compact cardinals have witness objects of rank 1: for all colourings, there exists a homogeneous set in \( \mathrm{V}_{\kappa + 1} \).
        \item Measurable cardinals have witness objects of rank 2: there is a \( \kappa \)-complete nonprincipal ultrafilter on \( \kappa \).
        The initial \( \forall x \) quantifier is not needed in this case.
        \item Surviving cardinals also have witness objects of rank 2, namely, a pair of ultrafilters.
    \end{enumerate}
\end{example}
In particular, inaccessibility is 0-stable, weak compactness is 1-stable, and measurability and survivability are 2-stable.
\begin{remark}
    If \( \beta \)-strong cardinals have witness objects, they cannot be of rank \( \beta \), because then they would reflect below.
    Witness objects for strength exist and are called \emph{extenders}, and if \( \mu \) is the least \( \beth \) fixed point larger than \( \abs{\mathrm{V}_{\kappa + \beta}} \), then the witness object for \( \beta \)-strength has rank at most \( \mu \).
\end{remark}
\begin{definition}
    A cardinal \( \kappa \) is called \emph{strong} if it is \( \beta \)-strong for all \( \beta < \lambda \).
\end{definition}
Importantly, the quantifiers are
\[ \forall \beta.\, \exists j.\, \mathrm{V}_{\kappa + \beta} \subseteq M \]
This does not say that there exists an embedding where all of the \( \mathrm{V}_{\kappa + \beta} \) are subsets of the same \( M \).
This notion cannot have a single witness object of a fixed rank, since otherwise, strength would reflect strength.

\subsection{Removing the inaccessible}
The ultrapower constructions used an inaccessible cardinal above a measurable cardinal, so that we could obtain a set-sized universe containing a measurable cardinal.
When trying to do this with the real universe, we encounter several problems.
\begin{enumerate}
    \item The definition of ultrapowers requires a set model.
    \item In the fundamental theorem of measurable cardinals, we have a quantification over \( j \) and \( M \).
    If these are proper classes, this quantification cannot be expressed in the usual language of set theory.
    \item Also, in the fundamental theorem of measurable cardinals, we use the notion of an elementary embedding, which is only definable for set models.
\end{enumerate}
To solve problem (i), we would like to construct \( \faktor{\mathrm{V}^\kappa}{\sim_U} \).
Note that \( \mathrm{V}^\kappa \) is a well-defined class; it is the class of all functions with domain \( \kappa \).
For such functions, it is easy to define the equivalence relation \( \sim_U \).
However, the equivalence classes \( [f]_U = \qty{g \in \mathrm{V}^\kappa \mid f \sim_U g} \) are all proper classes.
So \( \faktor{\mathrm{V}^\kappa}{\sim_U} \) is no longer a standard class; classes containing proper classes are typically not allowed.
This can be resolved using \emph{Scott's trick}.
If \( C \) is a nonempty class, then there is a minimal \( \alpha \) such that \( C \cap \mathrm{V}_\alpha \neq \varnothing \).
This is a nonempty set.
Define \( \operatorname{scott}(C) = C \cap \mathrm{V}_\alpha \) for this \( \alpha \).
Hence, if \( [f]_U \neq [g]_U \), we have \( \operatorname{scott}([f]_U) \neq \operatorname{scott}([g]_U) \).
We can therefore define
\[ \faktor{\mathrm{V}^\kappa}{U} = \qty{\operatorname{scott}([f]_U) \mid \dom f = \kappa} \]
To obtain our model \( M \), we took the Mostowski collapse of \( \faktor{{\mathrm{V}_\lambda}^\kappa}{U} \).
Therefore, we need a class version of the Mostowski collapse.
Recall that a relation \( E \subseteq C \times C \) is \emph{set-like} if for all \( x \in C \), the class \( \qty{y \in C \mid y \mathrel{E} x} \) is a set.
\begin{theorem}
    Let \( C \) be a class, and let \( E \subseteq C \times C \) be a binary relation on \( C \) that is well-founded, extensional, and set-like.
    Then there is a unique transitive class \( T \) such that \( (T, \in) \cong (C, E) \).
\end{theorem}
This may be proven in an almost identical fashion to Mostowski's collapsing theorem for sets.

For problems (ii) and (iii), recall that the fundamental theorem of measurable cardinals was that \( \mathsf{M}(\kappa) \) is equivalent to the statement that there is an elementary embedding \( j : \mathrm{V}_\lambda \to M \) with critical point \( \kappa \).
Measurability is witnessed at \( \kappa + 2 \), but the elementary embedding is not witnessed anywhere below \( \lambda \), so we cannot extend this definition to the usual universe.
We can solve this by extending our set theory to an appropriate class theory.
Standard class theories include \emph{von Neumann--Bernays--G\"odel} or \( \mathsf{NBG} \), and \emph{Morse--Kelley} or \( \mathsf{MK} \).
These theories have very different notions of class.
\( \mathsf{NBG} \) set theory is based upon the idea that definable formulas give the classes.
It is a `minimal class theory' where all classes are definable.
\( \mathsf{MK} \) is based on the idea that \( \mathrm{Ord} \) behaves externally like in inaccessible cardinal.
In this theory, there could be undefinable classes, and more classes than sets.

This resolves problem (ii), as we are permitted to work in a language in which we may quantify over proper classes.
However, this does not solve problem (iii).
Elementarity cannot be expressed as a single formula, but becomes a schema.
This causes additional problems as we need the existential over \( j \) and \( M \) to be part of each formula.
This could be solved by extending the language to add symbols for \( j \) and \( M \).
Another resolution is to observe that \( \Sigma_1 \)-elementarity suffices, as is explored in Kanamori's book \emph{The Higher Infinite} on page 45.
This can be defined using a single formula, therefore solving problem (iii).

\subsection{Supercompact cardinals}
\begin{definition}
    \( M \) is \emph{closed under \( \mu \)-sequences} if \( M^\mu \subseteq M \).
\end{definition}
\begin{theorem}
    If \( \kappa \) is measurable and \( j : \mathrm{V}_\lambda \to M \) is the ultrapower embedding, then \( M \) is closed under \( \kappa \)-sequences but not \( \kappa^+ \)-sequences.
\end{theorem}
\begin{proof}
    Let \( S = \qty{(f_\alpha) \mid \alpha < \kappa} \in M^\kappa \).
    We must show that \( S \in M \).
    Find \( h \) such that \( (h) = \kappa \).
    For \( \xi \in \kappa \), define \( g(\xi) \) to be a function with domain \( h(\xi) \) such that for all \( \alpha \in h(\xi) \),
    \[ g(\xi)(\alpha) = f_\alpha(\xi) \]
    Then
    \[ \qty{\xi \mid \dom g(\xi) = h(\xi)} = \kappa \in U \]
    By \L{}o\'s' theorem, \( \dom (g) = (h) = \kappa \).
    Further,
    \[ \qty{\xi \mid \forall \alpha \in \dom g(\xi).\, g(\xi)(\alpha) = f_\alpha(\xi)} = \kappa \in U \]
    so again by \L{}o\'s' theorem, if \( \alpha \in \dom (g) = \kappa \), then \( (g)(\alpha) = (f_\alpha) \).
    Hence \( (g) = S \).

    Let
    \[ T = \qty{j(\alpha) \mid \alpha < \kappa^+} \in M^{\kappa^+} \]
    We claim that \( T \notin M \).
    To prove this, we first show that \( T \) is unbounded in \( j(\kappa^+) \), which is equal to \( j(\kappa)^+ \) by elementarity.
    Indeed, consider an arbitrary \( (f) < j(\kappa^+) \).
    Then \( j(\kappa^+) = (c_{\kappa^+}) \), so without loss of generality we can assume \( f : \kappa \to \kappa^+ \).
    As \( \kappa^+ \) is regular, \( f \) is bounded by some \( \alpha < \kappa^+ \), so \( f : \kappa \to \alpha \).
    Then \( (f) < (c_\alpha) = j(\alpha) \in T \).

    Now, note that \( j(\kappa^+) = j(\kappa)^+ \) is a regular cardinal, so cannot have small unbounded subsets.
    But \( \abs{T} = \kappa^+ < j(\kappa)^+ \), so \( T \notin M \).
\end{proof}
\begin{definition}
    An embedding \( j \) is called \emph{\( \mu \)-supercompact} if \( M^\mu \subseteq M \).
    A cardinal \( \kappa \) is called \emph{\( \mu \)-supercompact} if there is a \( \mu \)-supercompact embedding with critical point \( \kappa \).
\end{definition}
Therefore, the theorem above shows that if \( \kappa \) is measurable, then it is \( \kappa \)-supercompact, and the ultrapower embedding is not \( \kappa^+ \)-supercompact, although there could be other embeddings that are.
\begin{definition}
    A cardinal \( \kappa \) is called \emph{supercompact} if it is \( \kappa \)-supercompact for all \( \mu < \lambda \).
\end{definition}
As with strong cardinals, the quantifiers are in the order
\[ \forall \mu.\, \exists j.\, M^\mu \subseteq M \]
If \( \kappa \) is \( 2^\kappa \)-supercompact, then \( \kappa \) is \( 2 \)-strong.
First note that \( \mathrm{V}_{\kappa + 2} = \mathcal P(\mathrm{V}_{\kappa + 1}) \) and \( \abs{\mathrm{V}_{\kappa + 1}} = \abs{\mathcal P(\mathrm{V}_\kappa)} = 2^\kappa \).
Every \( A \in \mathrm{V}_{\kappa + 1} \) is a \( 2^\kappa \)-sequence of elements of \( M \), so if every \( 2^\kappa \)-length sequence lies in \( M \), then \( A \in M \) as required.
In general, if \( \kappa \) is \( \abs{\mathrm{V}_{\kappa + \beta}} = \beth_{\kappa + \beta} \)-supercompact, then \( \kappa \) is \( (\beta + 1) \)-strong.
In particular,
\begin{corollary}
    Every supercompact cardinal is strong.
\end{corollary}

\subsection{The upper limit}
We now consider reversing the quantifier order in the definition of a strong cardinal.
\begin{definition}
    A cardinal \( \kappa \) is called \emph{Reinhardt} if there is an embedding \( j \) such that for all \( \beta \), we have \( \mathrm{V}_{\kappa + \beta} \subseteq M \), or equivalently, \( M = \mathrm{V}_\lambda \).
    In other words, there is an elementary embedding \( j : \mathrm{V}_\lambda \to \mathrm{V}_\lambda \) with critical point \( \kappa \).
\end{definition}
\begin{theorem}[Kunen]
    \( \mathsf{ZFC} \) proves that there are no Reinhardt cardinals.
\end{theorem}
It is an open problem whether \( \mathsf{ZF} \) without \( \mathsf{AC} \) proves there are no Reinhardt cardinals.
\begin{proof}
    Suppose \( j : \mathrm{V}_\lambda \to M \) has critical point \( \kappa \).
    Find the least \( j \)-fixed point above \( \kappa \), by defining
    \[ \kappa_0 = \kappa;\quad \kappa_{i + 1} = j(\kappa);\quad \hat \kappa = \bigcup_{i \in \omega} \kappa_i \]
    so \( j(\hat \kappa) = \hat \kappa \).
    We will show that \( \mathrm{V}_{\hat\kappa + 1} \nsubseteq M \), which is a result called \emph{Kunen's lemma}.
    This contradicts the assumption that \( M = \mathrm{V}_\lambda \).

    We need a combinatorial lemma due to Erd\H{o}s and Hajnal.
    For a cardinal \( \delta \), we say that \( f : [\delta]^\omega \to \delta \) is \emph{\( \omega \)-J\'onsson} if for every \( X \subseteq \delta \) such that \( \abs{X} = \delta \), we have \( \qty{f(A) \mid A \in [X]^\omega} = \delta \).
    The lemma states that every cardinal \( \delta \) has an \( \omega \)-J\'onsson function.

    Suppose that \( \mathrm{V}_{\hat\kappa + 1} \subseteq M \).
    Let \( f : [\hat\kappa]^\omega \to \hat\kappa \) be an \( \omega \)-J\'onsson function for \( \hat\kappa \).
    Then \( M \) believes that \( j(f) \) is an \( \omega \)-J\'onsson function for \( j(\hat\kappa) = \hat\kappa \).
    Define
    \[ X = \qty{j(\alpha) \mid \alpha \in \hat\kappa} \in \mathrm{V}_{\hat\kappa + 1} \]
    We claim that \( X \notin M \), finishing the proof.
    Suppose \( X \in M \).
    Clearly \( \abs{X} = \hat\kappa \), so then \( M \vDash \abs{X} = \hat\kappa \).
    We can apply the definition of an \( \omega \)-J\'onsson function in \( M \), which shows that
    \[ M \vDash \qty{j(f)(A) \mid A \in [X]^\omega} = \hat\kappa \]
    Any \( A \in [X]^\omega \) is of the form \( \qty{j(\alpha_i) \mid i \in \omega} \) for some \( a = \qty{\alpha_i \mid i \in \omega} \in [\hat\kappa]^\omega \).
    Then \( j(a) = \qty{j(\alpha_i) \mid i \in \omega} = A \).

    In general, if \( g(x) = y \), then \( g \) is a function, \( x \in \dom g \), and \( \langle x, y \rangle \in g \).
    Applying \( j \), we have that \( j(g) \) is a function, \( j(x) \in \dom j(g) \), and \( \langle j(x), j(y) \rangle \in j(g) \).
    So \( j(g)(j(x)) = j(y) = j(g(x)) \).
    Therefore, we obtain \( j(f)(A) = j(f)(j(a)) = j(f(a)) \).
    But \( f(a) \in \hat\kappa \), so \( j(f(a)) \in X \).
    Therefore,
    \[ M \vDash \hat\kappa = \qty{j(f)(A) \mid A \in [X]^\omega} \subseteq X \]
    But this cannot be true, for example because \( \kappa \notin X \) but \( \kappa \in \hat\kappa \).
\end{proof}
\begin{remark}
    \begin{enumerate}
        \item The combinatorial lemma was proven using \( \mathsf{AC} \), and it is not known whether the proof works without it.
        \item To prove Kunen's lemma, we did not need that \( \lambda \) is inaccessible.
        More explicitly, if \( j : \mathrm{V}_\alpha \to M \) is an elementary embedding with critical point \( \kappa \) such that \( \hat\alpha + 2 \leq \alpha \) (to guarantee that \( f \in \mathrm{V}_\alpha \)), then \( \mathrm{V}_{\hat\kappa + 1} \nsubseteq M \).
    \end{enumerate}
\end{remark}
\begin{corollary}
    For any ordinal \( \delta \), there is no elementary embedding \( j : \mathrm{V}_{\delta + 2} \to \mathrm{V}_{\delta + 2} \) with critical point less than \( \delta + 2 \).
\end{corollary}
\begin{proof}
    Observe that if \( \kappa < \delta + 2 \) is the critical point, we cannot have \( \kappa = \delta \) or \( \kappa = \delta + 1 \), because \( \delta \) and \( \delta + 1 \) are definable in \( \mathrm{V}_{\delta + 2} \).
    Then \( j(\kappa) < \delta \), so by induction all of the iterated images of \( \kappa \) under \( j \) are less than \( \delta \), so \( \hat\kappa \leq \delta \).
    Thus \( \hat\kappa + 2 \leq \delta + 2 \), so by remark (ii), \( \mathrm{V}_{\hat\kappa + 1} \nsubseteq \mathrm{V}_{\delta + 2} \), giving a contradiction.
\end{proof}
The axiom stating the existence of an analogous \( j : \mathrm{V}_{\delta + 1} \to \mathrm{V}_{\delta + 1} \) is called \( I1 \), and the existence of \( j : \mathrm{V}_\delta \to \mathrm{V}_\delta \) is called \( I3 \); there is an axiom \( I2 \) in between.
Clearly, if \( j : \mathrm{V}_{\delta + 1} \to \mathrm{V}_{\delta + 1} \) is elementary, then so is \( \eval{j}_{\mathrm{V}_\delta} : \mathrm{V}_\delta \to \mathrm{V}_\delta \), so \( I1 \) implies \( I3 \).
It has been hypothesised that \( I1 \) and \( I3 \) are inconsistent, but we do not yet have a proof.
